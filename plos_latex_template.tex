% Template for PLoS
% Version 3.6 Aug 2022
%
% % % % % % % % % % % % % % % % % % % % % %
%
% -- IMPORTANT NOTE
%
% This template contains comments intended 
% to minimize problems and delays during our production 
% process. Please follow the template instructions
% whenever possible.
%
% % % % % % % % % % % % % % % % % % % % % % % 
%
% Once your paper is accepted for publication, 
% PLEASE REMOVE ALL TRACKED CHANGES in this file 
% and leave only the final text of your manuscript. 
% PLOS recommends the use of latexdiff to track changes during review, as this will help to maintain a clean tex file.
% Visit https://www.ctan.org/pkg/latexdiff?lang=en for info or contact us at latex@plos.org.
%
%
% There are no restrictions on package use within the LaTeX files except that no packages listed in the template may be deleted.
%
% Please do not include colors or graphics in the text.
%
% The manuscript LaTeX source should be contained within a single file (do not use \input, \externaldocument, or similar commands).
%
% % % % % % % % % % % % % % % % % % % % % % %
%
% -- FIGURES AND TABLES
%
% Please include tables/figure captions directly after the paragraph where they are first cited in the text.
%
% DO NOT INCLUDE GRAPHICS IN YOUR MANUSCRIPT
% - Figures should be uploaded separately from your manuscript file. 
% - Figures generated using LaTeX should be extracted and removed from the PDF before submission. 
% - Figures containing multiple panels/subfigures must be combined into one image file before submission.
% For figure citations, please use "Fig" instead of "Figure".
% See http://journals.plos.org/plosone/s/figures for PLOS figure guidelines.
%
% Tables should be cell-based and may not contain:
% - spacing/line breaks within cells to alter layout or alignment
% - do not nest tabular environments (no tabular environments within tabular environments)
% - no graphics or colored text (cell background color/shading OK)
% See http://journals.plos.org/plosone/s/tables for table guidelines.
%
% For tables that exceed the width of the text column, use the adjustwidth environment as illustrated in the example table in text below.
%
% % % % % % % % % % % % % % % % % % % % % % % %
%
% -- EQUATIONS, MATH SYMBOLS, SUBSCRIPTS, AND SUPERSCRIPTS
%
% IMPORTANT
% Below are a few tips to help format your equations and other special characters according to our specifications. For more tips to help reduce the possibility of formatting errors during conversion, please see our LaTeX guidelines at http://journals.plos.org/plosone/s/latex
%
% For inline equations, please be sure to include all portions of an equation in the math environment.  For example, x$^2$ is incorrect; this should be formatted as $x^2$ (or $\mathrm{x}^2$ if the romanized font is desired).
%
% Do not include text that is not math in the math environment. For example, CO2 should be written as CO\textsubscript{2} instead of CO$_2$.
%
% Please add line breaks to long display equations when possible in order to fit size of the column. 
%
% For inline equations, please do not include punctuation (commas, etc) within the math environment unless this is part of the equation.
%
% When adding superscript or subscripts outside of brackets/braces, please group using {}.  For example, change "[U(D,E,\gamma)]^2" to "{[U(D,E,\gamma)]}^2". 
%
% Do not use \cal for caligraphic font.  Instead, use \mathcal{}
%
% % % % % % % % % % % % % % % % % % % % % % % % 
%
% Please contact latex@plos.org with any questions.
%
% % % % % % % % % % % % % % % % % % % % % % % %

\documentclass[10pt,letterpaper]{article}
\usepackage{textgreek}

% amsmath and amssymb packages, useful for mathematical formulas and symbols
\usepackage{amsmath,amssymb}
\usepackage{newunicodechar}
\newunicodechar{α}{\alpha}
\newunicodechar{≥}{\geq{}}
\newunicodechar{Δ}{\Delta}
\DeclareUnicodeCharacter{03B2}{\textbeta}
\newunicodechar{₂}{$_2$}
\newunicodechar{ }{~}
\newunicodechar{ }{\,}

% Use adjustwidth environment to exceed column width (see example table in text)
\usepackage{changepage}

% textcomp package and marvosym package for additional characters
\usepackage{textcomp,marvosym}

% cite package, to clean up citations in the main text. Do not remove.
\usepackage{natbib} % for \citep and \citet
% \usepackage{cite} % comment out or remove if using natbib

% Use nameref to cite supporting information files (see Supporting Information section for more info)
\usepackage{nameref,hyperref}

% line numbers
\usepackage[right]{lineno}

% ligatures disabled
\usepackage[nopatch=eqnum]{microtype}
\DisableLigatures[f]{encoding = *, family = * }

% color can be used to apply background shading to table cells only
\usepackage[table]{xcolor}
\usepackage{enumitem}
\usepackage{booktabs}

% array package and thick rules for tables
\usepackage{array}

% create "+" rule type for thick vertical lines
\newcolumntype{+}{!{\vrule width 2pt}}

% create \thickcline for thick horizontal lines of variable length
\newlength\savedwidth
\newcommand\thickcline[1]{%
  \noalign{\global\savedwidth\arrayrulewidth\global\arrayrulewidth 2pt}%
  \cline{#1}%
  \noalign{\vskip\arrayrulewidth}%
  \noalign{\global\arrayrulewidth\savedwidth}%
}

% \thickhline command for thick horizontal lines that span the table
\newcommand\thickhline{\noalign{\global\savedwidth\arrayrulewidth\global\arrayrulewidth 2pt}%
\hline
\noalign{\global\arrayrulewidth\savedwidth}}


% This is for Supplementary
\usepackage{graphicx}      % already in almost every template
\usepackage{subcaption}    % gives sub-figures + \subref
\usepackage{placeins}      % lets us slam a barrier before the SI
\usepackage[margin=1in]{geometry}

% For table
\usepackage{booktabs}
\usepackage{multirow}
\usepackage{array}
\usepackage{tabularx}
\usepackage{makecell} % in your preamble

% ---------- helper to switch counters to S-numbers ----------
\newcommand{\beginsupplement}{%
  \setcounter{table}{0}%
  \renewcommand{\thetable}{S\arabic{table}}%
  \setcounter{figure}{0}%
  \renewcommand{\thefigure}{S\arabic{figure}}}

% 2) Define a macro \beginsupplement that:
%    • Resets the figure/table counters
%    • Prefixes future figures/tables with “S”
\usepackage{etoolbox} % for \pretocmd and \setcounter


% Remove comment for double spacing
%\usepackage{setspace} 
%\doublespacing

% Text layout
\raggedright
\setlength{\parindent}{0.5cm}
\textwidth 5.25in 
\textheight 8.75in

% Bold the 'Figure #' in the caption and separate it from the title/caption with a period
% Captions will be left justified
\usepackage[aboveskip=1pt,labelfont=bf,labelsep=period,justification=raggedright,singlelinecheck=off]{caption}
\renewcommand{\figurename}{Fig}

% Use bibliography
\usepackage[numbers]{natbib}

% Remove brackets from numbering in List of References
\makeatletter
\renewcommand{\@biblabel}[1]{\quad#1.}
\makeatother

% links
\usepackage{hyperref}


% Header and Footer with logo
\usepackage{lastpage,fancyhdr,graphicx}
\usepackage{epstopdf}
%\pagestyle{myheadings}
\pagestyle{fancy}
\fancyhf{}
%\setlength{\headheight}{27.023pt}
%\lhead{\includegraphics[width=2.0in]{PLOS-submission.eps}}
\rfoot{\thepage/\pageref{LastPage}}
\renewcommand{\headrulewidth}{0pt}
\renewcommand{\footrule}{\hrule height 2pt \vspace{2mm}}
\fancyheadoffset[L]{2.25in}
\fancyfootoffset[L]{2.25in}
\lfoot{\today}

%% Include all macros below

\newcommand{\lorem}{{\bf LOREM}}
\newcommand{\ipsum}{{\bf IPSUM}}



%% END MACROS SECTION
\usepackage{graphicx}
\usepackage[aboveskip=1pt,labelfont=bf,labelsep=period,justification=raggedright,singlelinecheck=off]{caption}
\usepackage{placeins}

\begin{document}
\vspace*{0.2in}

% Title must be 250 characters or less.
\begin{flushleft}
{\Large
\textbf\newline{Sorghum Lipidomics Database} % Please use "sentence case" for title and headings (capitalize only the first word in a title (or heading), the first word in a subtitle (or subheading), and any proper nouns).
}
\newline
% Insert author names, affiliations and corresponding author email (do not include titles, positions, or degrees).
\\
Nirwan Tandukar\textsuperscript{1,2\Yinyang},
Ruthie Stokes\textsuperscript{3},
Name4 Surname\textsuperscript{2},
Name5 Surname\textsuperscript{2\ddag},
Name6 Surname\textsuperscript{2\ddag},
Rubén Rellán Álvarez\textsuperscript{1,3*},

\bigskip
\textbf{1} Department of Genetics and Genomics, North Carolina State University, Raleigh, NC, USA
\\
\textbf{2} Department of Bioinformatics, North Carolina State University, Raleigh, NC, USA
\\
\textbf{3} Department of Molecular and Structural Biochemistry,  North Carolina State University, Raleigh, NC, USA
\\
\bigskip


% Insert additional author notes using the symbols described below. Insert symbol callouts after author names as necessary.
% 
% Remove or comment out the author notes below if they aren't used.
%
% Primary Equal Contribution Note
\Yinyang These authors contributed equally to this work.

% Additional Equal Contribution Note
% Also use this double-dagger symbol for special authorship notes, such as senior authorship.
\ddag These authors also contributed equally to this work.

% Current address notes
\textcurrency Current Address: Dept/Program/Center, Institution Name, City, State, Country % change symbol to "\textcurrency a" if more than one current address note
% \textcurrency b Insert second current address 
% \textcurrency c Insert third current address



% Group/Consortium Author Note
\textpilcrow Membership list can be found in the Acknowledgments section.

% Use the asterisk to denote corresponding authorship and provide email address in note below.
* correspondingauthor@institute.edu

\end{flushleft}
% Please keep the abstract below 300 words
\section*{Abstract}
SAP lines



\linenumbers

% Use "Eq" instead of "Equation" for equation citations.
\section*{Introduction}

\subsection*{Lipid remodelling under abiotic constraints}

Plants remodel their membranes in a highly‐orchestrated manner when temperature or nutrient supply is sub‑optimal.  Below we summarise the characteristic fingerprints for \textbf{cold}, \textbf{phosphorus} and \textbf{nitrogen} stress, with emphasis on (i) class ratios that can be used as diagnostic indicators and (ii) individual molecular species that act as markers in lipidomic data sets.

%--------------------------------------------------------------------
\subsubsection*{Cold stress}
\label{sec:cold}

\begin{enumerate}[label=\textbf{\arabic*.}, leftmargin=1.2em]
  \item \textbf{Higher acyl‑chain unsaturation.}  Cold‐tolerant genotypes accumulate poly‑unsaturated fatty acids—principally 18\,:3, 18\,:2 and 18\,:1—leading to a higher double‑bond index (DBI) and preventing membrane rigidification at low temperature \citep[pp.~431–440, 460]{Low_temp_stress_Bhattacharya}.  An increase in DBI is consistently reported in tolerant lines of \textit{Arabidopsis}, maize and peanut \citep[pp.~11–12]{Lipid_transcriptome_Cold_stress_Yu}.

  \item \textbf{Class‑level reshaping.}  
        \begin{itemize}
          \item Poly‑unsaturated PC, PE, PG, MGDG and DGDG species rise, whereas their saturated counterparts decline \citep[pp.~3–4]{Low_temperatures_Wang,Low_temp_stress_Bhattacharya}.  
          \item The bilayer/non‑bilayer ratio, \(\mathrm{(PC+DGDG)/(PE+MGDG)}\), increases, stabilising the lamellar phase of membranes during freezing events \citep[pp.~492–493]{Low_temp_stress_Bhattacharya}.  
          \item Phosphatidic acid (PA) and lysophospholipids (LPC, LPE) surge, reflecting activation of phospholipase D and A, respectively \citep[pp.~456, 472--474]{Low_temp_stress_Bhattacharya}.
        \end{itemize}

  \item \textbf{Species‑level markers.}  In maize, PA\,36:5, PA\,36:6, DAG\,36:5 and DAG\,36:6 are elevated, whereas MGDG\,36:5 and multiple PC species decline \citep[pp.~6–8]{cold_tolerance_maize_Shi}.  Tolerant cultivars show higher TAG and lower DAG/TAG ratios compared with sensitive lines \citep[pp.~11]{Lipid_transcriptome_Cold_stress_Yu}.

  \item \textbf{Lipid signalling.}  PLD- and PLA‑derived PA and lyso‑lipids act as second messengers, triggering cold‐responsive gene networks \citep[pp.~454–456]{Low_temp_stress_Bhattacharya}.

  \item \textbf{Functional outcome.}  Increased unsaturation and altered bilayer propensity maintain a fluid–crystalline phase, securing electron transport and nutrient transport across membranes at low temperature \citep[pp.~463–465]{Low_temp_stress_Bhattacharya}.
\end{enumerate}

%--------------------------------------------------------------------
\subsubsection*{Phosphorus deprivation}
\label{sec:phosphorus}

\begin{enumerate}[label=\textbf{\arabic*.}, leftmargin=1.2em]
  \item \textbf{Phospholipid depletion.}  Major phospholipids (PC, PE, PG, PI, PS, PA) decline sharply as they serve as an internal Pi source; in soybean leaves every phospholipid class decreased under Pi limitation \citep[pp.~1,\,3,\,5]{lipid_remodeling_low_P_Saito}.

  \item \textbf{Compensatory rise of non‑P lipids.}  MGDG, DGDG, SQDG and the diagnostic glucuronosyldiacylglycerol (GlcADG) accumulate to preserve membrane surface area \citep[pp.~3--4]{Phosphate_deficiency_Wang}.  GlcADG can increase up to 14‑fold in soybean \citep{lipid_remodeling_low_P_Saito}.

  \item \textbf{Diagnostic ratio.}  The phospholipid/galactolipid ratio (PL/GL) drops from \(\sim\)0.3 (P‐sufficient) to \(\le 0.05\) under severe P stress in field‐grown camelina \citep[page~4]{Phosphate_deficiency_Wang}.

  \item \textbf{Tissue specificity.}  Older leaves are remodelled first, exporting Pi to developing tissues \citep[pp.~1,\,5]{lipid_remodeling_low_P_Saito}.

  \item \textbf{Enzymatic drivers.}  Phospholipase C/D hydrolyse PC and PE; MGDG/DGDG and SQDG synthases are up‑regulated to supply the replacement lipids \citep[pp.~1–2, 6]{Phosphate_scaracity_Xue}.
\end{enumerate}

%--------------------------------------------------------------------
\subsubsection*{Nitrogen deprivation}
\label{sec:nitrogen}

\begin{enumerate}[label=\textbf{\arabic*.}, leftmargin=1.2em]
  \item \textbf{Chloroplast glycolipids.}  Rapeseed shows an 18 % (leaf) to 35 % (root) reduction in MGDG; DGDG declines by 23 % in roots, resulting in a suppressed \(\mathrm{MGDG/DGDG}\) ratio \citep[pp.~5--9]{nitrogen_deficiency_lipid_Yang}.

  \item \textbf{Phospholipid curtailment.}  PC, PE, PI, PS and PA all decrease markedly, the latter by more than 90 % in both organs \citep{nitrogen_deficiency_lipid_Yang}.

  \item \textbf{Storage lipids.}  TAG remains unchanged in rapeseed but accumulates in mature tea leaves under low N, suggesting carbon re‑allocation from photosynthetic (N‑rich) to storage pools \citep[pp.~6--7]{Nitrogen_fertilizer_Ruan}.

  \item \textbf{Integrated carbon‑nitrogen balance.}  Lower nitrogen leaves a surplus of assimilated carbon; plants divert it into TAG or into highly unsaturated MGDG 36:5/36:6 species observed in tea shoots at high N \citep{Nitrogen_fertilizer_Ruan}.
\end{enumerate}

%--------------------------------------------------------------------
\subsubsection*{Synthesis}

Cold, P and N stress each trigger a distinctive yet overlapping pattern of lipid remodelling:

\begin{itemize}
  \item \textbf{Cold} prioritises \emph{unsaturation} and bilayer‑to‑non‑bilayer balance to maintain fluidity.  
  \item \textbf{Pi starvation} reallocates phosphorus by replacing phospho‑lipids with galacto‑ and sulfo‑lipids, sharply lowering the PL/GL ratio.  
  \item \textbf{N starvation} down‑regulates chloroplast glycolipids and phospholipids, sometimes storing excess carbon as TAG.  
\end{itemize}

These shifts are mirrored in our sorghum data: unsaturation indices rise under early low‑temperature planting; the \(\mathrm{DGDG/MGDG}\) and \(\mathrm{SQDG/PG}\) ratios increase under P‑limited, low‑input conditions; and TAG/PC as well as \(\mathrm{TG/DG}\) ratios escalate when available nitrogen is low (see Sections \ref{sec:cold}, \ref{sec:phosphorus} and \ref{sec:nitrogen}).

- Stress in plants specifically in sorghum

- Cold stress

- low Nitrogen

- low Phosphorus

- Relate to climate change?



%---------------------------------------------------------------
\begin{table}[ht]
\centering
\small
\setlength{\tabcolsep}{6pt}
\renewcommand{\arraystretch}{1.15}
\begin{tabular}{@{}p{2.3cm} p{4.2cm} p{1.3cm} p{4.5cm} p{1.7cm}@{}}
\toprule
\textbf{Stress} & \textbf{Key lipid class / molecular species} & \textbf{Direction\textsuperscript{a}} & \textbf{Diagnostic (ratio) or remark} & \textbf{Ref.} \\
\midrule
\multirow{6}{*}{\textbf{Cold}} 
 & Poly‑unsaturated FA (18:3, 18:2, 18:1)            & $\uparrow$ & Higher double‑bond index (DBI)                                & \citet{Low_temp_stress_Bhattacharya} \\
 & Unsat.\ PC, PE, PG, MGDG, DGDG                     & $\uparrow$ & Bilayer lipids enriched                                        & \citet{Low_temperatures_Wang}        \\
 & PA (incl.\ PA\,36:5;\,36:6)                        & $\uparrow$ & PLD activation; signalling                                     & \citet{cold_tolerance_maize_Shi}     \\
 & LPC, LPE                                           & $\uparrow$ & PLA activity                                                   & \citet{Low_temp_stress_Bhattacharya} \\
 & DAG\,36:5;\,36:6                                   & $\uparrow$ & Mobilisation of PC unsat.\ chains                             & \citet{cold_tolerance_maize_Shi}     \\
 & TAG (total)                                        & $\uparrow$ & \textit{cf.}\ DAG/TAG $\downarrow$ in tolerant lines           & \citet{Lipid_transcriptome_Cold_stress_Yu} \\
 \cmidrule{2-5}
 & \multicolumn{2}{@{}l}{\textit{Cold ratios}}       & (PC\,+\,DGDG)/(PE\,+\,MGDG)\:$\uparrow$; \ DAG/TAG\:$\downarrow$ & \citet{Low_temp_stress_Bhattacharya} \\
\midrule
\multirow{5}{*}{\textbf{P deficiency}} 
 & PC, PE, PG, PI, PS, PA                             & $\downarrow$ & Release of Pi pool                                            & \citet{lipid_remodeling_low_P_Saito} \\
 & MGDG, DGDG                                          & $\uparrow$  & Galacto‑lipid replacement                                     & \citet{Phosphate_deficiency_Wang}    \\
 & SQDG                                               & $\uparrow$  & Sulfo‑lipid substitution                                      & \citet{Phosphate_deficiency_Wang}    \\
 & GlcADG                                             & $\uparrow$  & Pi‑stress biomarker (14‑fold)                                 & \citet{lipid_remodeling_low_P_Saito} \\
 & \multicolumn{2}{@{}l}{\textit{P ratios}}           & PL/GL $\downarrow$ (to $\le$ 0.05); DGDG/MGDG $\uparrow$        & \citet{Phosphate_deficiency_Wang}    \\
\midrule
\multirow{5}{*}{\textbf{N deficiency}} 
 & MGDG (leaf, root)                                  & $\downarrow$ & 18–35 \% reduction                                            & \citet{nitrogen_deficiency_lipid_Yang} \\
 & DGDG (root)                                        & $\downarrow$ & 24 \% reduction                                               & \citet{nitrogen_deficiency_lipid_Yang} \\
 & PC, PE, PI, PS, PA                                 & $\downarrow$ & PA $\downarrow$ > 90 \%                                       & \citet{nitrogen_deficiency_lipid_Yang} \\
 & TAG (mature tea leaves)                            & $\uparrow$  & Carbon sink under low N                                       & \citet{Nitrogen_fertilizer_Ruan}      \\
 & \multicolumn{2}{@{}l}{\textit{N ratios}}           & MGDG/DGDG $\downarrow$; TAG/PC $\uparrow$; TG/DG $\uparrow$     & \citet{nitrogen_deficiency_lipid_Yang} \\
\bottomrule
\multicolumn{5}{l}{\footnotesize \textsuperscript{a}\,$\uparrow$ increase, $\downarrow$ decrease relative to control or sufficient nutrient.}
\end{tabular}
\caption{Core lipid markers and class ratios characterising cold, phosphorus and nitrogen stress as distilled from the literature survey.  Arrows indicate the direction of change in stressed tissues.}
\label{tab:lipid_markers}
\end{table}
%---------------------------------------------------------------


\subsection*{OPLS-DA}
4. Supervised multivariate discrimination of the lipidomes
4.1 What OPLS‑DA does—in plain language
Orthogonal‑Projection to Latent Structures Discriminant Analysis (OPLS‑DA) is a supervised extension of Principal‑Component Analysis that forces the first latent component to explain only variance that is correlated with a user‑defined class vector (here, Control vs Low‑input). Any systematic variation that is orthogonal to class membership—batch differences, genotype heterogeneity, stochastic noise—is captured in subsequent, “orthogonal” components.
The outcome is a model that

separates classes as strongly as possible along a single predictive axis (t1),

isolates uninformative variance on orthogonal axes (to1, to2, …), and

provides a Variable‑Importance in Projection (VIP) score for every lipid, ranking its contribution to class separation.

Because the method is supervised, we rigorously guard against over‑fitting by (i) k‑fold cross‑validation (Q²) and (ii) permutation testing.

4.2 Model quality and diagnostic overview (Fig. 6a–d)
<div align="center"><em>Insert composite “overview” panel here</em></div>
Panel (a) – Component summary 
The model contains one predictive component (p1) and two orthogonal components (o1, o2). Predictive component p1 alone explains 94 % of the class variance (R²Y) and 94 % of its cross‑validated predictability (Q²Y), well above the commonly accepted 0.5 threshold (grey reference line). Orthogonal components capture residual variation in the lipid matrix (R²X ≈ 0.70 in total) that is unrelated to treatment.

Panel (b) – Permutation test 
Two‑hundred random permutations of the class vector were fitted to the same data (grey dots). None of the permuted models approaches the real model’s Q² or R² (black bars on the right). The probability of obtaining an equal or better model by chance is pR²Y = 0.05; pQ² = 0.05, confirming that the discrimination is not an over‑fit artefact.

Panel (c) – Observation diagnostics 
Score‑distance (SD, leverage) is plotted against orthogonal distance (OD, residual variance). Dashed lines denote the 95 % Hotelling T² limits. Only five genotypes (labels s441, s736, s447…) exceed one or both thresholds; visual inspection of chromatograms revealed no technical issue, hence they were retained.

Panel (d) – Score plot 
Each point represents a genotype; blue = Control, red = Low‑input. The two classes form well‑separated, compact clouds along the predictive axis t1 (39 % of total lipid variance), while the orthogonal axis to1 (30 %) captures within‑class dispersion. The 95 % confidence ellipse encloses every sample except the mild outliers identified in panel (c). Together with the high Q², this demonstrates a robust lipidomic signature of the Low‑input treatment.

4.3 Discriminatory lipids revealed by VIP analysis (Fig. 6e)
<div align="center"><em>Insert VIP bar plot (top 30) here</em></div>
VIP values quantify how strongly each lipid contributes to the predictive component. A conservative cut‑off of VIP > 1.3 (dashed line) yielded 28 discriminatory species (Supplementary Table S7). The top ten are displayed in Fig. 6e and encompass several structural classes:

Rank	Lipid (annotation)	Class	VIP	Change in Low‑input†
1	Nostoxanthin	Terpenoid	1.88	 ↑ 4.1‑fold
2	Triethylene‑glycol bis(2‑ethyl‑hexanoate)	Plasticizer ‡	1.82	 ↓ 3.6‑fold
3	ε‑Decalactone	Organic compound	1.79	 ↑ 2.9‑fold
…	…	…	…	…

† Fold‑change refers to median linear intensity.
‡ Likely exogenous contaminant; retained for completeness but excluded from biological discussion.

Biologically meaningful drivers include:

Triacyl‑glycerols TG(56:6), TG(46:0) – consistently enriched under Low‑input, supporting the nitrogen‑remobilisation hypothesis (Fig. 6e, pale‑blue bars).

Phosphatidylethanolamine PE(34:1) – depleted in Low‑input, aligning with the phospholipid‑to‑glycolipid replacement model under combined low P/low N.

β‑Sitosterol – a sterol known to modulate membrane order, markedly increased (VIP 1.55), coherent with the observed membrane‑integrity ratios (Fig. 5).

These VIP‑identified species therefore corroborate and extend the univariate ratio analysis.

4.4 Interpretation
The OPLS‑DA model demonstrates that field Low‑input treatment imprints a strong, coherent lipidomic signature across 350–380 sorghum genotypes, explaining > 80 % of class variance with excellent predictability.

Discriminating lipids belong to storage (TG), membrane (PE) and signalling (sterols, terpenoids) pools, suggesting a coordinated adjustment of carbon and nutrient allocation.

The very limited number of statistical outliers and the stringent permutation validation underline the robustness of the result.

Together, these findings establish a quantitative link between agronomic low‑input management and the sorghum lipidome, and highlight specific lipid species that can serve as biomarkers for future breeding or physiological studies.

\section*{Materials and methods}

\subsection*{Plant Material and Growth Conditions}
In our research, we utilized the Sorghum Association Panel (SAP), consisting of 400 accessions designed to encompass extensive genetic and phenotypic diversity. This collection includes both temperate-adapted breeding lines and tropical landraces. The accessions represent the five official botanical races—bicolor, caudatum, durra, guinea, and kafir capturing a range of domestication and subsequent adaptation events.

SAP was first genotyped using simple sequence repeat markers, followed by low-coverage genotyping by sequencing (GBS). For a more comprehensive variation set, Boatwright et al. resequenced all entries using whole genome sequencing (WGS) with an average depth of 38× (ranging from 25–72×). The variant data from WGS revealed approximately 43.98 million polymorphisms, including roughly 38 million SNPs, 5 million small insertions/deletions, and around 17×10\^5 copy-number variants. Notably, about 50\% of the 5 kb genomic windows displayed variants identifiable solely through the WGS dataset, highlighting WGS's significantly superior genomic coverage compared to GBS. While GBS variants were predominantly located in genic regions, the WGS data were more evenly distributed across genic and intergenic regions.

The analysis of population structure divided the panel into six genetic groups that reflected both the botanical races and the primary breeding categories. Genome-wide linkage disequilibrium decreased to half its maximum around 20 kb, although there were deviations specific to each chromosome. The consequent high-density variant map, along with the well-documented structure, establishes the resequenced SAP as a valuable tool for examining diversity and conducting whole-genome wide association studies (GWAS).

We evaluated SAP accessions across two different field settings during two consecutive growing seasons (2019 and 2022) at the Pee Dee Research and Education Center, Clemson University, Florence, South Carolina. The "control" condition, herein denoted as C, involved standard agronomic inputs with sufficient levels of nitrogen (N) and phosphorus (P) along with a typical planting schedule. In contrast, the "low input" condition, herein denoted as LI,  featured reduced N and P coupled with earlier planting to mimic a cold environment.

\subsection*{Lipid Extraction}
CONTACT RUTHIE STOKES



\subsection*{Lipidomics  Raw Data Processing}
At first, the raw peak intensity signals were processed to eliminate any features with a retention time under 1 minute. Next, the intensity columns at the sample level were extracted and each was relabeled by extracting the run number and "PI" identifier from the original LC-MS filenames. Subsequently, these columns were organized in ascending run order to ensure that the downstream matrices accurately represent the chronological sequence of injections.

Subsequently, blank filtering was conducted by determining the lowest signal value for each feature among all biological samples, as well as calculating the average signal for the blank injections. Features where the lowest sample value was under ten times the average of the blanks were excluded. Following this, all blank injections were discarded, ensuring that only authentic sample peaks remained for further analysis.

For quality control, the leftover “QC” injections were extracted from the filtered table and analyzed using summary statistics and box plots to ensure consistent signals across batches (see Supplementary Fig. 1). Any features or runs displaying clear outlier patterns in these diagnostics were marked for exclusion from further analyses.

Following the cleaning process, the data were organized for SERRF normalization (Systematic Error Removal Using Random Forest) (ref). The initial two rows were designated for sample and QC labels. Subsequently, run numbers and batch identifiers, sourced from the mass spectrometer batch-run mapping file, were appended. The formatted CSV was then submitted to the SERRF server (https://slfan2013.github.io/SERRF-online/\#) to obtain the normalized output. After applying SERRF, only biological samples were preserved. Lipid features exhibiting over 50\% zero values were excluded. Any remaining zeros were substituted with two-thirds of the minimum nonzero value for that feature to prevent potential infinite logarithmic transformations.

Finally, we conducted an additional quality control step specifically aimed at eliminating any  spatial patterns across our experimental trials. This was achieved using the R package \texttt{SpATS} \citep{Rodriguez-Alvarez2018}, which applies a two-dimensional P-spline ANOVA surface over the field coordinates. For every lipid feature, we characterized its intensity as
\begin{align}
  y_{ij} &= \mu + f_{\mathrm{row}}(i) + f_{\mathrm{col}}(j) + f_{\mathrm{row,col}}(i,j) + \varepsilon_{ij},
\end{align}

where \(f_{\mathrm{row}}\) and \(f_{\mathrm{col}}\) represent smooth functions that model systematic effects across rows and columns, respectively, while \(f_{\mathrm{row,col}}\) is a smooth interaction surface that handles more complex spatial gradients. The residuals, defined as the difference between observed intensity and the fitted spatial trend, were utilized as our "cleaned" phenotype values. This methodology effectively corrects for positional artifacts, such as edge effects, which could interfere with subsequent analyses. Detailed smoothing parameters, including the number of knots, penalty orders, and comprehensive model specifications, can be found in our GitHub repository at \texttt{scripts/spats\_qc.R}.

\subsection*{Lipid Annotation}

\subsubsection*{Lipid Identification and Quantification}
The normalized intensities of detected lipid species were organized into conventional lipid classes and subclasses (see Supplementary Table~1). In each sample, the total intensity for a class was obtained by summing the intensities of the species within that class. To manage variability in signal intensity due to different runs or injections, these class totals were normalized relative to the total ion current (TIC) of the sample. As a result, the relative abundances were presented as percentages of the TIC by adding up intensities across all lipid classes in the sample.

The calculated percentages were employed to generate composition bar charts. For each main class and its subclasses, we determined the TIC fraction for each sample and then averaged these percentages across samples for each condition (Control, $n=384$; LowInput, $n=362$). Lipids were categorized into traditional and non‐traditional classes. Traditional lipid classes consisted of core glycerolipids and glycerophospholipids such as triacylglycerols (TG), diacylglycerols (DG), monoacylglycerols (MG), phosphatidylcholines (PC), phosphatidylethanolamines (PE), phosphatidylinositols (PI), digalactosyldiacylglycerols (DGDG), monogalactosyldiacylglycerols (MGDG), sulfoquinovosyldiacylglycerols (SQDG), sphingomyelins (SM), alkyl‐ether glycerophospholipids (AEG), lysophosphatidylcholines (LPC), lysophosphatidylethanolamines (LPE), phosphatidylglycerols (PG), phosphatidic acids (PA), and phosphatidylserines (PS). Non-traditional lipid classes included compounds such as terpenoid and pigment‐related molecules recognized by accurate‐mass MS but residing outside the main membrane glycerolipid framework(refer to Supplementary Table 1 for the full list). 

\subsubsection*{Lipid Ratio Identification}
To determine the key lipid ratios most significantly influenced by the shift C LI, we utilized the cumulative class-level abundances of lipids and calculated all possible pairwise ratios. These calculations were then analyzed through orthogonal partial least squares–discriminant analysis (OPLS-DA) employing the ropls package in R (v3.3.2). Within each model, R²X and R²Y represented the proportion of variance explained in the predictor (lipid ratios) and response (sample class) matrices, respectively, and Q² evaluated the model's predictive capability. To avoid overfitting and verify the statistical significance of our OPLS-DA models, a permutation test was conducted for each predictive component.

To identify the most distinguishing ratios, we analyzed the Variable Importance in Projection (VIP) scores generated by OPLS-DA. A threshold of 1.3 for VIP was used. These high-ranking ratios not only enhanced the multivariate differentiation between C and LI samples but also linked to key stress-related membrane remodeling pathways known for their association with cold and nutrient stress, emphasizing both their statistical reliability and biological relevance.

\subsubsection*{Lipid Ratio Calculation and Statistical Test}

To quantify condition-specific shifts between lipid classes, we worked directly with 
\textt{log\textsubscript{10}-ratios of class-level relative abundances}.  
The normalization was done as follows:

\begin{enumerate}
  \item \textbf{Per-sample TIC normalization}.  
        For each sample, the raw peak intensities of all detected lipid species were summed (\emph{total-ion current}, TIC). The intensity of each species was divided by the TIC of that sample, which yielded a relative abundance (\(\mathrm{Intensity}/\mathrm{TIC}\)).
  \item \textbf{Log\textsubscript{10} transformation with pseudo-count}.  
        Due to the fact that many relative abundances are either minuscule or zero, half of the smallest nonzero value in the sample (\(\varepsilon\)) was added to each species. Subsequently, \(\log_{10}(x+\varepsilon)\) was calculated.
        This stabilizes the variance.
  \item \textbf{Class-level aggregation (mean log\textsubscript{10})}.  
        Lipid species were categorized into both traditional classes, such as PC, PE, DGDG, MGDG, TG, and DG, as well as non-traditional categories like Steroid and Terpenoid (refer to Supp Table 1). For each sample, the \(\log_{10}\) values of every species within a class were calculated and averaged, resulting in a single \emph{class\_log} value for each sample and class: \[
          \text{class\_log}_{i,c}=\frac{1}{n_c}\sum_{k\in c}\log_{10}
            \bigl(\tfrac{\mathrm{Intensity}_{i,k}}{\mathrm{TIC}_i}+\varepsilon_i\bigr).
        \]
\end{enumerate}

Pairwise comparisons, referred to henceforth as \emph{log-ratios}, were subsequently calculated by taking straightforward differences between class\_log values:\[
  \mathrm{DGDG\_PC} = \text{class\_log}_{\mathrm{DGDG}} -
                      \text{class\_log}_{\mathrm{PC}},
  \quad
  \mathrm{TG\_DG}   = \text{class\_log}_{\mathrm{TG}} -
                      \text{class\_log}_{\mathrm{DG}},
  \ldots
\]
Subtraction on the log scale is algebraically equivalent to a
fold-change, each metric is the \(\log_{10}\) ratio of two class abundances
(e.g.\ \(\log_{10}\tfrac{\text{DGDG}}{\text{PC}}\)).
Positive values therefore indicate enrichment of the numerator class relative
to the denominator, and vice versa.

Each log-ratio metric involved a comparison between C and LI samples conducted via Welch's two-sample \(t\)-test (\texttt{t.test(Value ~ Condition)} in \textsf{R}), noted for its robustness against differences in sample sizes and variances. The \(p\)-values obtained were displayed on violin and box plots, using standard notation: *** for \(p<0.001\), ** for \(p<0.01\), * for \(p<0.05\). Detailed test statistics and adjusted \(p\)-values are available in the github repository.


\subsection*{Principal Component Analysis (PCA)}  
We performed three complementary PCA workflows in R.  First, we ran PCA on the \emph{individual} lipid species abundances (log–TIC normalized).  Second, we summed abundances by \emph{lipid class} (e.g.\ TG, DG, PC, MGDG, SQDG) and repeated PCA to highlight macro-scale shifts in broad functional pools.  Third, we computed key \emph{class–ratio} metrics (e.g.\ galactolipid / phospholipid, TG / DG) and carried out PCA on these derived traits to focus on balance–based reprogramming under nutrient limitation. In all cases, data were mean–centered and unit–scaled prior to analysis.  For each PCA, we retained the first two principal components for visualization with \texttt{ggplot2}, added 95\% confidence ellipses around experimental groups, and overlaid loading vectors to identify the lipid species, classes, or ratios driving the greatest variance.  Scree plots were inspected to confirm that PC1 + PC2 captured the majority of structured variation, and breakpoints in the loading magnitudes guided our interpretation of key biochemical modules.  


\subsection*{Genome-wide Association Studies (GWAS)} 
Separate GWAS analyses were carried out for each lipid trait and under each field condition via the mixed linear model (MLM) featured in GEMMA (v2.3) (ref). To address population structure and unseen relatedness, a centered relatedness matrix (kinship) was computed from SNP genotype data. For each lipid trait, the MLM was applied using the kinship matrix to handle population stratification effects. Besides individual trait examinations, lipids were categorized by biochemical class (refer to Supplementary Table 1), and principal component analysis (PCA) was applied in each class to identify primary variation directions; subsequently, independent GWAS were conducted on the first two PCs for each class. For all association tests we applied a stringent significance threshold of \(p < 10^{-7}\) (i.e., \(-\log_{10}(p) \geq 7\)) to account for multiple comparisons.

\subsection*{Gene Annotation}  
Single nucleotide polymorphisms (SNPs) were aligned with the Sorghum bicolor reference genome v3.1 (BTx623). For each marker, a 50 kb segment was designated, spanning 25 kb on either side, and all gene models within this area were retrieved. Functional annotations and homology insights were obtained from Phytozome (https://phytozome.jgi.doe.gov), SorghumBase (https://sorghumbase.com), and TAIR for corresponding Arabidopsis thaliana orthologs. Genes with known roles in N, P, cold tolerance, or lipid metabolism were specifically selected. We aggregated the frequency of each candidate gene within all lipid GWAS findings and marked those with the highest recurrence showing -log10(p-values) of 7 or greater.

\subsection*{Orthogonal Projections to Latent Structures Discriminant Analysis (OPLS-DA)}

This study employed Orthogonal Projections to Latent Structures Discriminant Analysis (OPLS-DA) to identify and classify lipidomic profiles across various experimental conditions. OPLS-DA includes an orthogonal signal correction phase compared to Partial Least Squares Discriminant Analysis (PLS-DA). This phase distinguishes the variation in the predictor matrix \(X\) that is specifically associated with class membership (\(Y\)) from the unrelated variation. By partitioning \(X\) into predictive and orthogonal dimensions, OPLS-DA enhances the model's clarity and reduces confounding effects arising from variability within the same class.

The OPLS-DA model decomposes \(X\) according to the equation
\[
  X = T_{p} P_{p}^{T} \;+\; T_{o} P_{o}^{T} \;+\; E,
\]
where
\begin{itemize}
  \item \(T_{p}\) and \(P_{p}\) are the predictive score and loading matrices capturing variation correlated with \(Y\),
  \item \(T_{o}\) and \(P_{o}\) are the orthogonal score and loading matrices capturing structured variation orthogonal to \(Y\),
  \item \(E\) is the residual matrix representing unexplained variation.
\end{itemize}


The analysis was restricted to 15 lipid classes that were accurately identified (TG, DG, MG, PC, PE, PI, DGDG, MGDG, SQDG, SM, AEG, LPC, LPE, PA, PS). To highlight compositional shifts within samples and minimize variability arising from total signals, TIC was used. In order to study the compositional relationships between two distinct environments, lipid ratios were employed. All possible pairwise ratios between the mean log-relative abundances of classes were calculated. The OPLS-DA approach was utilized for all the ratios using the \texttt{ropls} package (v1.34.0) on the matrix of ratio features, \(\mathbf{X}\), and the response vector \(Y\) (C versus LI). A single predictive component (\texttt{predI = 1}) was stipulated, and cross-validation was used for the selection of orthogonal components (\texttt{orthoI = NA}). Features automatically underwent mean-centering and unit-variance scaling by \texttt{ropls} (\texttt{scaleC = "standard"}). Model performance was assessed using cross-validated \(R^{2}_{Y}\) and \(Q^{2}\), and lipid ratios that functioned as discriminators were identified via Variable Importance in Projection (VIP) scores (\(\text{VIP} > 1\)) which were used for further analysis. 

In order to reduce the risk of overfitting, a seven-fold cross-validation (CV) approach with balanced class distributions was employed. Within each CV cycle, the dataset was partitioned into seven segments; six segments were used for model training, while the remaining segment was assigned for evaluating predictive performance. We report the \(R^{2}_{Y}\) (explained variance of \(Y\)) and \(Q^{2}\) (predictive capability), averaged across the folds. To validate that the computed \(Q^{2}\) exceeded the 95th percentile of permuted \(Q^{2}\) values, a permutation test consisting of 200 permutations of the \(Y\) labels was conducted.


\subsection*{Hierarchical Clustering}
For each lipid and genotype, we performed a hierarchical clustering using the \texttt{hclust} function in R.  Before clustering, the lipid intensities were scaled to zero mean and unit variance.  A Euclidean distance matrix was calculated, and the groups were merged according to Ward's minimum variance criterion (Ward.D2).  The dendrograms were rendered with the \texttt{dendextend} package, and two-dimensional clustergrams were generated using the \texttt{ clustergram.R} script.  


\subsection*{Random Forest Regression and TreeSHAP–Based Feature Importance Analysis}
For the purposes of this analysis, raw intensities were subjected to a log\textsubscript{10} transformation and subsequently median-centered. Flowering time (FT) measured in days and plant height (PT) measured in centimeters were assessed from the field of cultivation, with all instances of missing values duly excluded. Genetic principal components (PCs) were derived from SNP genotypes within the Sorghum Association Panel by employing standard principal component analysis (PCA) routines facilitated by the SNPRelate package in R. The principal five PCs, which encapsulate the population structure, were integrated with the lipid and phenotype datasets. Thereafter, both the phenotypic variables and each lipid feature were adjusted for ancestry effects through linear regression residualization based on these PCs.\[
\begin{aligned}
P_i &= \alpha + \sum_{m=1}^{M} \beta_m\,PC_{i m} + \varepsilon_i,
\quad \widetilde{P}_i = \varepsilon_i,\\
L_{k,i} &= \alpha_k + \sum_{m=1}^{M} \gamma_{k m}\,PC_{i m} + \eta_{k,i},
\quad \widetilde{L}_{k,i} = \eta_{k,i}.
\end{aligned}
\]
where \(P_i\) is the phenotype (e.g.\ flowering time, plant height) of line \(i\), 
\(L_{k,i}\) is the abundance of lipid \(k\) in line \(i\), 
\(PC_{i m}\) are the top \(M\) genetic principal components for line \(i\), 
and \(\varepsilon_i\) and \(\eta_{k,i}\) are the residuals (structure‑corrected phenotype and lipid values, respectively).


Residualized lipids were employed as predictors while residualized P functioned as the response variable within a Random Forest regression framework, facilitated by the \texttt{ranger} package. The dataset was partitioned into 80\% training data and 20\% testing data through stratification based on k-means clusters of the principal components, ensuring the preservation of genetic structure. Hyperparameter (\texttt{mtry}, \texttt{min.node.size}, \texttt{sample.fraction}) optimization was conducted through Bayesian optimization implemented in \texttt{tuneRanger}, utilizing the out-of-bag RMSE as the optimization criterion. The model's final performance was assessed on the reserved test data set using metrics including RMSE, MAE, Pearson’s $r$, $R^2$, bias, and normalized RMSE.

In order to evaluate the contribution of each lipid to the predictions of P, the TreeSHAP algorithm was employed using the \texttt{treeshap} package. The trained \texttt{ranger} forest was amalgamated with the training dataset, and exact Shapley values $\phi_{i,k}$ were subsequently calculated for each sample $i$ and lipid $k$. The global importance was defined as the mean absolute Shapley value $\frac{1}{N}\sum_i|\phi_{i,k}|$, resulting in a prioritized list of candidate lipids. The most significant features were depicted using bar charts and beeswarm plots illustrating the per-sample Shapley value distributions.

\subsection*{Pathway Enrichment Analysis}

\subsection*{Data Availability}
Data processing and statistical analyzes were performed in R (version 4.3.3) using. All the codes, figures, and pipeline are described in the GitHub repository: github.com/nirwan1265/SoLD\_paper.

% Results and Discussion can be combined.
\section*{Results}

\subsection*{Quality Control and Signal Normalization of Lipidomics Data}
Instrument performance was measured using total ion current (TIC) traces throughout the injection sequence for batches C (Supplementary Fig. \ref{fig:S1}A) and LI (Supplementary Fig. \ref{fig:S1}B). As expected, the blanks (gray) remain near zero TIC, the internal standards (green) cluster tightly around their nominal signal, and the quality controls (red) track reproducibly throughout the run. The injections of samples (blue) exhibit the highest TIC with no isolated outliers or sudden jumps. These profiles confirm that instrument performance was stable over time, with consistent sensitivity and no evidence of progressive signal decay or unexpected artifacts.

After normalization of the SERRF for instrumental artifacts, in sets C and LI, the relative standard deviation (RSD) of the sample  for all lipid characteristics decreased from 3. 86 \% and 1. 18 \% (raw data)  to 0. 94 \% and 0. 51 \%, respectively,  (Supplementary Figs. \ref{fig:S2}A and \ref{fig:S2}B, top panel). This reduction in technical variability demonstrates that SERRF effectively removes batch-related effects, producing more consistent peak areas across injections. Similarly, PCA of the signal pre- and post-SERRF shows that the points cluster tightly, indicating that most of the remaining variance is biological rather than instrumental (Supplementary Figs. \ref{fig:S2}A and \ref{fig:S2}B, bottom panel).

SPATS analysis. 



\subsection*{Overview of lipid count}
The comprehensive enumeration of various lipid species and their respective classes identified under Control (C) and Low-Input (LI) conditions is presented in Supplementary Figures \ref{fig:S4}, \ref{fig:S5}, and \href{https://docs.google.com/spreadsheets/d/1SB90-QLYheKEzmHCUIh1UfgkrtbL064s8Oo5BfwFaV0/edit?gid=0#gid=0}{Supplementary Table S1}. Triacylglycerols (TG) show the greatest diversity in species, with 71 distinct species identified in C conditions compared to 69 under LI conditions. Monoacylglycerols (MG) reveal a significant reduction, decreasing from eight species in C to five in LI conditions. Phosphatidylcholines (PC), sulfoquinovosyldiacylglycerols (SQDG), and diacylglycerols (DG) each experience a reduction of exactly one species under LI. In contrast, phosphatidylethanolamines (PE) exhibit an increase, with 15 species identified under Low-Input conditions as opposed to 14 in Control conditions.

Upon aggregating by class, the overall composition of the lipidome remained largely stable, despite variations in lipid species. The C condition yields 226 species, whereas the LI  yields 224. Glycerolipids constitute the largest class, comprising 120 species in the C and 114 in the LI , followed by glycerophospholipids, with 60 and 64 species, respectively. All other classes, terpenoids (12 and 14), fatty acids (7 and 11), sphingolipids (9 and 5), ether lipids (7 each), prenols (4 each), sterols (2 and 3), and betaine lipids (1 for LI), collectively represent only a minor portion of the species (Supplementary Fig \ref{fig:S4}). These observations indicate that, notwithstanding significant environmental stresses in the form of N, P, and cold, the LI workflow maintains nearly the entire range and balance of lipidome species.

\subsection*{Overview of lipid diversity}

The distribution of the total-ion-current (TIC) indicates that glycerolipids dominate the lipid signal in both conditions constituting 64.4 \% of the TIC in the standard C run, but decreasing to 52.3 \% under the low-input (LI) conditions (Supplementary Figure \ref{fig:S5}A). Glycerophospholipids also experience a decline in their TIC contribution, from 30.2 \% to 25.3 \%, whereas sphingolipids, which are nearly undetectable in the C analysis (0.2\%), exhibit a significant increase to 17.3 \% of the TIC under LI. Minor lipid classes, including sterols and betaine lipids (< 0.1–0.2 \%), fatty acids (which reduce from 0.7 \% to 0.3 \%), ether lipids (approximately 1 \%), and terpenoids (rising from 3.4 \% to 3.8 \%) collectively comprise the remaining 5–7 \% in both conditions.


Upon examination of the glycerolipid pool (Supplementary Figure \ref{fig:S5}B), monogalactosyldiacylglycerols constitute the predominant subclass (comprising 50.6\% of glycerolipid TIC in C, and 47.3\% in LI), followed by diacylglycerols (increasing from 20.3\% to 21.9\%) and triacylglycerols (rising from 4.0\% to 7.9\%). In contrast, both monoacylglycerols (decreasing from 2.0\% to 1.7\%) and sulfoquinovosyldiacylglycerols (declining from 4.3\% to 2.0\%) exhibit slight reductions under conditions of nutrient and cold stress, whereas glycolipid subclasses such as MGDG and DGDG remain constant.

Within the glycerophospholipids (Supplementary Figure \ref{fig:S5}C), phosphatidylcholines dominate at almost 80 \% of the class in both workflows (79.9 \% vs 76.9 \%), phosphatidylethanolamines edge up from 18.4 \% to 20.8 \%, and phosphatidylglycerols increase from 1.3 \% to 1.5 \%. Lysophospholipids (LPC, LPE) and rare headgroups (PS, PA) remain below half a percent.

LI results in alterations in certain subclass proportions, most notably the relative increase in sphingolipids and a marginal rise in triacylglycerols. However, the fundamental structure of the lipidome, spanning from major classes to head group families, remains consistently preserved under the LI.


\subsection*{Overview of individual lipid species}
We computed, for each lipid species, the mean percentage contribution within each species (the relative abundance of each species divided by the total abundance of that species under a given condition, averaged between samples). 

Supplementary Table~1 lists the species sorted by their mean percentage contribution under C and LI. In the following, we summarize the dominant species and any moderate compositional changes between the two conditions.

\bigskip
\textbf{Acyl‐ether glycerol (AEG)}  \\
AEG(o-32:3), AEG(o-34:4), and AEG(o-34:5) are the most prevalent species, together comprising more than 75\% of the overall AEG signal across both conditions. Importantly, AEG(o-32:3) experiences a notable increase under LI, going from 23.2\% to 35\%, whereas AEG(o-34:5) decreases from 26.9\% to 21.2\%. AEG(o-34:4) remains relatively constant. Minor species like AEG(o-36:6) and AEG(o-30:2) are present at low levels and exhibit minimal changes. In summary, the AEG pool composition undergoes a moderate adjustment under stress, with the significant rise in AEG(o-32:3) being a key factor in this alteration.

\textbf{Diacylglycerol (DG)}  \\
The dataset reveals a predominant presence of the diacylglycerol (DG) species DG(18:3/18:3), accounting for approximately 78.9 \% of the DG pool within the C and 73.7 \% under LI. The subsequent most prevalent species, DG(16:0/18:2) and DG(12:0/12:0), each comprise about 4–5 \% of the DG signal across both experimental conditions. DG(16:0/18:3) is observed to increase from 2.2 \% in the C condition to 5.2 \%, and DG(18:1/2:0) rises from a minimally detectable 0.4 \% to 6.3 \%, marking the most notable compositional changes within the DG class. By contrast, the majority of other species of mid-abundance such as DG(18:2/18:2), DG(16:0/18:1), DG(18:1/18:3), DG(18:2/18:3), DG(18:1/18:1), and DG(18:1/18:2) exhibit variations amounting to only a few tenths of a percent across different conditions. Several minor species, detectable during the C run, namely DG(16:0/16:0) at 1.0 \% and DG(16:0/2:0) and DG(18:0/18:0) at 0.4 \% each, decrease below the detection threshold or are absent under LI. In contrast, trace lipids such as DG(16:1/18:2), DG(18:0/20:3), and DG(8:0/8:0) are exclusively detectable in LI conditions at concentrations of less than 0.1 \%. These findings collectively suggest that despite the DG profile predominantly being defined by its 18:3/18:3 core, LI condition tends to enrich certain unsaturated and short-chain species.


\textbf{Digalactosyldiacylglycerols (DGDG)}  \\
Digalactosyldiacylglycerols (DGDG) are predominantly constituted by the species DGDG(18:2/18:4), which comprises 63.9 \% of the DGDG composition within the C, with a minor increase to 65.4 \% under LI. The second most prevalent species, DGDG(16:0/18:3), accounts for 27.6 \% in the C  compared to 24.2 \% in the LI, and is followed by DGDG(18:0/18:3), which exhibits an increase from 4\% to 5.5\%. Minor constituents such as DGDG(16:0/18:2) exhibit significant stability (2.1\% to 2.2\%), whereas DGDG(18:2/18:3) slightly increases from 1\% to 1.4\%. The low-abundance species DGDG(16:0/18:1) and DGDG(18:0/18:2) remain near the detection limit, with negligible changes from 0.9\% to 0.8\% and 0.4\% to 0.5\%, respectively. Collectively, these data indicate that the LI effectively maintains the DGDG composition, with only minor enrichment of fully saturated and monounsaturated species at the periphery.

\textbf{Monogalactosyldiacylglycerol (MGDG)} \\
The MGDG composition is predominantly characterized by the fully saturated MGDG(18:3/18:3), which increases from 87.3 \% of the monogalactolipids in the C to 89.3 \% under LI. The second most prevalent species, MGDG(16:0/18:3), decreases from 5.5 \% to 4.2 \%, while the di- and tri-unsaturated species MGDG(18:2/18:3) and MGDG(18:2/18:2) experience slight reductions (5 \% to 4.5 \% and 2.1 \% → 2.0 \%, respectively). The minor component, MGDG(16:0/18:1), remains a trace constituent (<0.1 \%) in both conditions. Thus, the MGDG profile preserves its distinctive polyunsaturated signature, with only a marginal increase in the abundance of the 18:3/18:3 species at the expense of those with moderate abundance.

\textbf{Lysophospholipids (LPC, LPE)} \\
Lysophospholipids constitute a minor component of the lipidome, predominated by LPC(16:0) and LPC(18:3). Within the C, LPC(16:0) comprises 73 \% of the lysophosphatidylcholines, with LPC(18:3) constituting the remaining 27 \%. Conversely, under LI, these proportions are altered to 62.6 \% and 30.5 \%, respectively. An additional presence of LPC(18:2) (6.9 \%) is exclusively identified in the LI, suggesting that LI may reveal trace unsaturated species. Across both conditions, lysophosphatidylethanolamine consists entirely of LPE(16:0), maintaining a constant 100 \% presence within the LPE pool, irrespective of the condition.

\textbf{Monoacylglycerol (MG)}  \\
Monoacylglycerols are predominantly characterized by the polyunsaturated MG(18:3), which increases from 51.8\% of the MG pool in the C to 58.2\% in LI. Concurrently, MG(18:1) shows an increase from 16.1\% to 18.4\%, and MG(12:0) increases from 12.4\% to 14.6\%, while the saturated MG(16:0) decreases from 8.9\% to 6.9\%. The di-unsaturated MG(18:2) also experiences a slight reduction (3.4\% to 2.0\%). Several minor species detectable in the C,MG(16:1) (1.1\%) and MG(18:0) (6.3\%), alongside trace amounts of MG(20:4) (<0.1\%), fall below the detection threshold or are absent under the LI conditions. Overall, shifts in the MG profile towards more highly unsaturated C18 species while reducing the presence of low-abundance saturated and monounsaturated forms are observed in the LI.

\textbf{Phosphatidic acid (PA)}  \\
PA(34:2) constitutes ~100\% of the PA pool in both conditions, indicating that only one species is reliably detected.

\textbf{Phosphatidylcholine (PC)} \\ 
Phosphatidylcholines remain dominated by the 16:0/18:2 species which drops from 37.9 \% of the PC pool to 27.7 \% . The overall PC landscape is noticeably reshaped under LI conditions. The saturated 16:0/18:0 species rises from 14.6 \% to 19.4 \%, while the mono-unsaturated 16:0/18:1 form falls from 8.9 \% to 4.8 \%. The polyunsaturated 16:0/18:3 remains roughly constant at ~8–9 \%. Among the C36 lipids, 16:0/20:5 declines modestly (7.5 \% to 5.7 \%) and 16:0/20:4 collapses (6.9 \% to 0.6 \%), whereas the previously negligible 16:1/18:2 species surges from 0.1 \% to 7.5 \%, and 18:0/18:1 likewise emerges from 0.1 \% to 5.8 \% of the PC signal. A new C38 variant, 18:2/20:0, climbs from below detection to 3.1 \%, and a handful of other minor PCs (e.g. 16:1/20:4 at 5.3 \%) appear only in the LI run. All other low-abundance forms long-chain di- and tri-unsaturated species, lysolysophosphatidylcholines and odd-chain variants—remain at or below 1 \% in both conditions. Thus, while PC(16:0/18:2) continues to dominate, LI selectively unmasks and enriches several minor PC isoforms and shifts the saturated/unsaturated balance across the class.


\textbf{Phosphatidylethanolamine (PE)}  \\
Under conditions of LI, the phosphatidylethanolamine (PE) landscape undergoes a significant transformation, emphasizing species with higher unsaturation levels. Within the standard C, PE(16:0/18:2) predominates, comprising 50.3\% of the PE pool. However, its representation diminishes markedly to 20.7\% under LI. Conversely, PE(16:0/18:3), which is polyunsaturated, escalates from a modest 5.4\% to 30.2\%, consequently becoming one of the most prevalent PEs under LI. In parallel, PE(16:0/20:5) nearly triples its proportion from 3.7\% to 9.5\%, while PE(14:0/22:6) and PE(18:0/18:2), which are either undetectable or minor in the C (at 0.4\% and 1.0\%, respectively), increase significantly to 8.5\% and 7.7\%. In contrast, di-unsaturated species such as PE(16:0/20:4) and PE(18:2/18:2) decrease from 12.1\% to 7.3\% and 8.4\% to 6.4\%, respectively, and the mono-unsaturated PE(16:0/18:1) experiences a drastic reduction from 10.2\% to merely 0.9\%. Minor species including PE(18:3/18:3) and PE(18:2/18:3) display only slight alterations, with several low-abundance forms either emerging or vanishing between analyses. Collectively, these findings indicate that LI preferentially reveals and enhances the presence of long-chain, polyunsaturated ethanolamines, while concurrently diminishing their mono- and diunsaturated counterparts.


\textbf{Phosphatidylglycerol (PG)}  \\
Phosphatidylglycerols (PG) undergo a significant compositional reorganization under LI. In the  C, PG(16:0/18:1) accounts for 60.3\% of the PG pool, yet its proportion decreases to 56.6\% in LI. Similarly, the fully saturated PG(16:0/16:0) experiences a reduction from 33.4\% in the C to merely 17.5\% in LI. Most markedly, PG(16:0/18:0), which is undetectable or absent in the C, emerges as a prominent component at 25.9\% under LI conditions, whereas its regioisomer PG(18:0/16:0), previously contributing 6.3\% in C, falls below the detection limit. Collectively, these variations indicate that the reduction of input materials favors the persistence of the unsaturated 34:1 species at the expense of the fully saturated 32:0 form, while concurrently revealing a previously undetected saturated 34:0 isomer in the Low-Input framework.

\textbf{Phosphatidylglycerol (PS)}  \\
In C, the partitioning of the PS pool is characterized by the predominance of two molecular species. PS(18:0/20:4) constitutes the majority at 65.1\%, while PS(18:0/18:2) accounts for the remaining 34.9\%. Upon LI condition, only PS(18:0/18:2) remains detectable, representing 100\% of the PS signal as PS(18:0/20:4) declines below the detection threshold. This transition demonstrates that the LI preferentially retains the 18:0/18:2 head-group configuration.


\textbf{Sphingomyelin (SM)}  \\
SM(35:1) comprises ~100\% of SM in both conditions.

\textbf{Sulfoquinovosyldiacylglycerol (SQDG)}  \\
Sulfoquinovosyldiacylglycerols experience a substantial compositional transformation under LI. In the C, SQDG(16:0/18:3) and SQDG(18:3/18:3) collectively account for 74.6\% of the composition (48.4\% and 26.2\%, respectively). However, in LI, the prevalence of these species declines significantly to 3.3\% and 4.6\%, respectively. Conversely, SQDG(16:1/18:3) increases from a negligible 0.4\% to 59.7\%, while SQDG(16:0/18:1) rises from 0.7\% to 24.4\%. The fully saturated SQDG(16:0/16:0) decreases from 15.6\% to 0.3\%. All other variants of moderate abundance—such as SQDG(18:1/18:3), SQDG(18:2/18:3), and SQDG(18:0/18:3) undergo modest changes to between 0.6\% and 4.1\%, and trace species like SQDG(16:0/14:0) remain below 1\%. Overall, in the LI condition, originally dominant species 16:0/18:3 and 18:3/18:3 replaces a new predominance of 16:1/18:3 and enriches the mono-unsaturated 16:0/18:1 species, thereby revealing a substantially different SQDG composition profile.

\textbf{Triacylglycerol (TG)}  \\
Under LI, the triglyceride (TG) profile experiences substantial reorganization, predominantly focusing on a singular polyunsaturated species: TG(16:1/20:1/20:2), which increases markedly from a mere 0.1\% of the C TG pool to 27.3\%. In contrast, the fully saturated TG(16:0/16:0/16:0) is significantly diminished, approximately halving from 18.4\% to 8.1\%, while the di-unsaturated TG(16:1/16:1/22:5) declines from 20.6\% to 5.3\%. Notably, TG(18:2/18:2/22:1), previously undetectable under C conditions, constitutes 15.8\% of the TG signal in the LI. Concurrently, the partially unsaturated TG(18:1/18:2/18:3) remains largely stable, registering a minor increase from 5.3\% to 5.9\%. However, all typically mid-abundance forms, TG(16:0/18:2/18:3), TG(18:2/18:2/18:4), TG(16:0/18:1/18:3), and TG(16:0/18:3/18:3) are reduced to near zero. A few minor species, such as TG(18:0/18:2/20:1) (increasing from <0.1\% to 3\%) and TG(12:0/12:0/14:0) (decreasing from 1.4\% to 0.6\%), exhibit directional shifts. Nonetheless, a salient insight emerges that the LI condition markedly alters the conventional TG framework, introducing a pronounced peak in the highly unsaturated TG(16:1/20:1/20:2), while also identifying TG(18:2/18:2/22:1) as a prominent new constituent.
\bigskip

In summary, the LI initiates a coordinated reprogramming of the plant lipidome, favoring neutral, energy-storage lipids and highly unsaturated membrane components while simultaneously restructuring the glycerophospholipid profile. Triacylglycerols, which typically exhibit saturated backbones, are almost entirely supplanted by a novel, polyunsaturated TG(16:1/20:1/20:2) peak (0.1\% to 27.3\%) and the emerging TG(18:2/18:2/22:1) (below detection to 15.8\%), even as the traditional TG(16:0/16:0/16:0) reduces its prevalence. Diacylglycerols remain predominantly composed of DG(18:3/18:3) (78.9\% to 73.7\%) but show enrichment in DG(16:0/18:3) and DG(18:1/2:0) under LI conditions. The monoglyceride pool similarly shifts towards its polyunsaturated C18 core (MG(18:3) 51.8\% to 58.2\%), and the galactolipids MGDG and DGDG maintain their highly unsaturated characteristics with only minor subclass variations. Sulfoquinovosyldiacylglycerols transition from SQDG(16:0/18:3) and SQDG(18:3/18:3) dominants to a predominant SQDG(16:1/18:3) majority (0.4\% to 59.7\%) with increased SQDG(16:0/18:1). Among lysophospholipids, LPC increasingly favors the saturated LPC(16:0) at the reduction of LPC(18:3), while LPE consistently remains LPE(16:0). Glycerophospholipids experience deliberate remodeling. Phosphatidylcholine (PC) reduces its 16:0/18:2 backbone (37.9\% to 27.7\%) in preference for 16:1/18:2 and 18:0/18:1 variations. Phosphatidylethanolamine (PE) significantly enriches in PE(16:0/18:3) (5.4\% to 30.2\%) and longer-chain polyunsaturates. Phosphatidylglycerol (PG) reveals a novel PG(16:0/18:0) isomer while decreasing PG(16:0/16:0). Phosphatidylserine (PS) converges to a singular PS(18:0/18:2) species. Collectively, these transformations highlight that LI not only preserves but in certain contexts amplifies the core unsaturated and energy-storage lipids, while revealing low-abundance variants and fundamentally redistributing membrane lipid classes.

\subsection*{Overview of lipid profiles}
The contribution of each lipid class to the total TIC under both C and LI conditions was quantified by calculating the average percent‐TIC across all C (\(n = 384\)) and all LI (\(n = 362\)) samples (Fig.~\ref{fig:Fig1_lipid_class}A). The most significant reduction under LI conditions was observed in the primary galactolipid MGDG (35.8\%~\(\rightarrow\)~33.5\%, \(\Delta = -2.3\%\)), followed by the glycerophospholipid PC (26.5\%~\(\rightarrow\)~25.2\%, \(\Delta = -1.3\%\)) and the sulfolipid SQDG (3.0\%~\(\rightarrow\)~1.8\%, \(\Delta = -1.2\%\)). In contrast, the largest absolute increase was noted in TG (2.4\%~\(\rightarrow\)~6.0\%, \(\Delta = +3.6\%\)), along with DG (8.9\%~\(\rightarrow\)~10.1\%, \(\Delta = +1.2\%\)), and a moderate rise in DGDG (13.2\%~\(\rightarrow\)~13.6\%, \(\Delta = +0.4\%\)) and PE (6.1\%~\(\rightarrow\)~6.5\%, \(\Delta = +0.4\%\)). All other species (AEG, Cer, GalCer, LPC/LPE, PA, PG, PS, FA) constituted less than 1\% of TIC or exhibited only minimal changes.

A PCA of individual lipid species was conducted, revealing an expected distinct separation between the C and LI samples, with the first and second principal components (PC1 and PC2) accounting for 29.6\% and 22.8\% of the variance, respectively (Fig. \ref{fig:Fig1_lipid_class}B). Each sample was plotted based on the log-transformed relative abundance of all detected lipid molecules. The C samples formed a tight cluster in the upper right quadrant, whereas the LI samples were positioned toward the lower left. This shift suggests a comprehensive reprogramming of lipid composition in response to nutrient stress. We also conducted a PCA of aggregated lipid classes, which also demonstrated a distinct separation (Fig. \ref{fig:Fig1_lipid_class}C). The first two principal components accounted for 44.8\% (PC1) and 24.9\% (PC2) of the total variance, respectively. The differentiation between conditions is primarily observed along PC2, with C samples clustering at the positive PC2 axis, and LI samples aligning at the negative axis. This distribution signifies an alteration in lipid composition under cold and nutrient stress. PC1 captures variations in membrane fluidity and carbon storage lipids. Lipids associated with photosynthesis such as SQDG, MGDG, DGDG, PG, PS, and FA exhibited a strong positive loading on PC2, positioning themselves in the upper quadrants and closely associating with the C condition. Conversely, phospholipids and storage lipids (e.g., PC, PE, PA, LPC, DG, MG, TG, GalCer) displayed a negative loading on PC2 (and a positive loading on PC1), projecting into the lower quadrants, in alignment with the LI condition. This trend suggests that under LI conditions, plants tend to reallocate membrane lipids towards turnover and neutral lipid storage, exemplified by TG, whereas, under C conditions, the maintenance of photosynthetic membrane integrity is preserved.

Under LI conditions, the lipidome transitions from a photosynthetically driven profile to a membrane restructuring and energy storage profile. Samples subjected to LI exhibit a marked reduction in galactolipids (MGDG decreased by 2.3\%) and crucial phospholipids and sulfolipids (PC decreased by 1.3\% and SQDG by 1.2\%), whereas TGs increase by 3.6\% and DGs by 1.2\%. At the classification level, glycerolipids (both galacto- and neutral lipids) increase from 68.1\% to 69.3\% of the TIC, while glycerophospholipids decrease from 31.8\% to 30.3\%. Principal Component Analysis (PCA) provides further support for these observations. The samples exhibit separation along PC2, contrasting plastidic, photosynthesis-associated lipids (such as SQDG, MGDG, DGDG, PG, and PS) with lipids associated with membrane turnover and storage (such as PC, PE, PA, LPC, DG, MG, TG, and GalCer). Biologically, this pattern reflects the degradation or repurposing of thylakoid membranes due to cold and low nutrient availability, alongside an increase in neutral lipid (TG) synthesis to serve as a carbon reservoir and an enhancement in phospholipid turnover for signaling under stress and membrane repair.


\begin{figure}[htbp]
  \centering
  \includegraphics[width=\textwidth]{fig/main/Fig1.png}

    \caption{Overview of lipidomics for Control and Lowinput \\
    \textbf{(A)} Percent‐of‐TIC breakdown for individual lipid \emph{species}, averaged across all C (n = 384) and LI (n = 362) samples. Only species whose mean contribution >= 3 \% are labeled in‐bar.  
    \textbf{(B)} PCA biplot of individual lipid species, with scores colored by condition and vectors showing species loadings. Each point represents a sample, projected based on the log-transformed relative abundance of all individual lipid molecules. Samples from the Control condition (purple) cluster in the upper-right quadrant, while those from the LowInput condition (yellow) shift toward the lower-left, indicating broad reprogramming of lipid composition under stress. The separation along PC1 (29.6\%) and PC2 (22.8\%) captures variance driven by both lipid abundance and diversity at the species level.
    \textbf{(C)} PCA biplot of summed lipid classes, with sample scores and class‐ratio loadings.}
    
  \label{fig:Fig1_lipid_class}
\end{figure}

\subsection*{Genome-wide Association Studies of Lipids Associated with Control and Lowinput}
We conducted GWAS with three distinct layers of data: (1) individual lipid species, which comprise both lipids  as well as other compounds (Supp Table 1), (2) aggregates of lipid class abundances, and (3) all conceivable pairwise ratios of these aggregated classes. Utilizing a significance threshold of $\log_{10}(p)\ge7$ for lipids and $\log_{10}(p)\ge5$ for other compounds , we identified W genes associated with lipid species (see Supplementary Table 2), X genes associated with other compounds (refer to Supplementary Table 3), Y genes associated with all of the summed lipid classes, and Z genes associated with the ratios of summed classes (refer to Supplementary Table 4) in a 25kb window. We conducted manual annotations for these candidate genes predicated on well-documented functions—such as lipid metabolism, N assimilation, P homeostasis, or response to cold stress. The full set of annotations is available in Supplementary Table 6. All the GWAS results can be obtained from the shiny app. Here, we explain some GWAS with important functions relating to the phenotype and the lowinput stresses. Fig \ref{fig:Fig3}


Re rin gwas:
PG(16:0/18:0)
\subsection*{DGAT1 Controls Triacylglycerol Storage in Response to Nitrogen Limitation and Cold}

We identified the gene \textit{SORBI\_3010G170000}, which encodes Acyl‐CoA:diacylglycerol acyltransferase 1 (DGAT1, analogous to Arabidopsis TG1), in five distinct GWASs: TG(58:3), TG(58:4), TG(58:5), TG(60:4), and TG(60:5). DGAT1 is responsible for the essential final conversion of DG into TG, which is a key lipid for carbon and energy storage in seeds and stress-affected vegetative tissues \cite{Zhang2009,Yang2011}. In Arabidopsis, low N levels result in the TG accumulation within leaves due to increased levels of DGAT1 and OLEOSIN1 \cite{Yang2011}. The ABA signaling pathway, involving the transcription factor ABI4, directly stimulates DGAT1 by interacting with CE1 elements (CACCG) in its promoter. In \emph{abi4} mutants, both DGAT1 stimulation and TG accumulation are reduced, emphasizing the significance of ABI4 during N deficiency \cite{Yang2011}. Additionally, DGAT1 is highly responsive to cold temperatures (4°C) and plays an essential role in freeze tolerance. Arabidopsis mutants deficient in \emph{dgat1} develop chlorosis and increased cell mortality under cold stress, with reduced TG but higher DG and PA levels \cite{Tan2018}. This elevated PA production induces RbohD-dependent ROS formation, causing oxidative stress. Increased DG kinase activity (DGK2/3/5) (GWAS results) further boosts PA, while the removal of \emph{dgk} genes restores cold tolerance, suggesting a balance between DGAT1 and DGK is essential for managing ROS and adapting to cold stress \cite{Tan2018}. In seeds, both DGAT1 and phospholipid:diacylglycerol acyltransferase 1 (PDAT1) are vital for optimal oil body development. \emph{dgat1} mutants have a 20–40\% decline in seed oil content (see Lipid annotation Section 5), whereas double mutants (\emph{dgat1/pdat1}) or RNAi lines demonstrate an 80\% decrease in TG, resulting in fertility and embryonic issues \cite{Zhang2009}. Overexpression of DGAT1 enhances seed weight and oil production, highlighting its crucial role in regulating TG levels throughout plant development \cite{Zhang2009,Yang2011}.


\subsection*{SQDG Metabolism and Its Role in Phosphate‐Starvation Responses}

In our GWAS of SQDG(32:0) levels, the top locus was \textit{SORBI\_3002G000600}, which encodes the plant ortholog of sulfoquinovosyltransferase (SQD2). SQDG is a negatively charged glycolipid (sulfoquinovose = 6‑deoxy‑6‑sulfonato‑glucose) that constitutes up to 10–20\% of chloroplast thylakoid lipids and is critical for stabilizing photosystem II, photosystem I, and cytochrome \emph{b}\(_6\)\emph{f} complexes \citep{Yu2002,Qin2015,Umena2011}.

Biosynthesis proceeds in two enzymatic steps \citep{Yu2002,Sun2021}:
\begin{enumerate}[label=(\arabic*)]
  \item \textit{SQD1} (UDP‑sulfoquinovose synthase): 
        \[
           \mathrm{UDP\!-\!Glc} + \mathrm{SO_3^{2-}} \;\longrightarrow\; \mathrm{UDP\!-\!sulfoquinovose}
        \]
  \item \textit{SQD2} (sulfoquinovosyltransferase): 
        \[
           \mathrm{UDP\!-\!sulfoquinovose} + \mathrm{diacylglycerol} \;\longrightarrow\; \mathrm{SQDG}
        \]
\end{enumerate}

Under phosphate (Pi) starvation, plants degrade phospholipids (e.g.\ PG) to recycle Pi, while \textit{SQD1} and \textit{SQD2} are transcriptionally upregulated, leading to increased SQDG accumulation and preservation of thylakoid membrane functions \citep{Essigmann1998,Nakamura2013,Sun2021}. In rice, \textit{OsPHR2} directly activates \textit{OsSQD1} under Pi deficiency, and loss of \textit{OsPHR2} impairs SQDG levels, alters fatty‐acid composition, and reduces photosynthetic efficiency \citep{Sun2021}. Similarly, \emph{sqd2} mutants in \emph{Arabidopsis thaliana} are unable to synthesize SQDG and exhibit growth defects under low‐Pi conditions, underscoring the essential role of SQDG in replacing anionic phospholipids in the chloroplast \citep{Yu2002}.


\subsection*{Alternative Oxidase Roles in Photoprotection and Nitrate Assimilation}

Through our GWAS focused on alpha-carotene, we  identified an alternative oxidase (AOX) gene. Alpha-carotene (\(\alpha\)-carotene), a secondary chloroplast carotenoid, is primarily located within the reaction centers of photosystem I (PSI) and photosystem II (PSII), with only minor quantities found in the peripheral light-harvesting complexes \citep{Young1989}. It bears structural similarity to \(\beta\)-carotene, absorbs blue-green light, and facilitates energy transfer to chlorophyll while concurrently quenching triplet chlorophyll and reactive oxygen species (ROS) to safeguard the photosynthetic apparatus from photooxidative damage under intense light stress. Its co-localization with \(\beta\)-carotene in pigment–protein complexes indicates a contributory role in stabilizing the core structures of PSII and PSI \citep{Young1989}. The mitochondrial AOX pathway offers a non-phosphorylating alternative to cytochrome oxidase, directly oxidizing ubiquinol to water, thereby preventing over-reduction of the photosynthetic electron transport chain \citep{Vishwakarma2015}. AOX1A, the dominant isoform in green tissues, plays a role in dissipating excess reducing equivalents produced by photosynthesis, supports non-photochemical quenching (NPQ), and collaborates with the chloroplast malate–oxaloacetate shuttle to sustain cellular redox homeostasis. Under conditions of stress, such as high light or drought, that inhibit the cytochrome pathway, AOX activity curbs ROS formation and maintains photosynthetic efficiency \citep{Vishwakarma2015}. In addition to its photoprotective function, AOX is vital for nitrate assimilation in plants. During NO\(_3^-\) reduction, the accumulation of reducing equivalents may lead to chloroplast over-reduction; AOX counters this by channeling excess reductants into mitochondrial respiration, thereby preventing oxidative stress and sustaining photosynthesis \citep{Gandin2014}. Studies involving \emph{aox1a} T-DNA insertion mutants in \emph{Arabidopsis thaliana} corroborate that AOX engages with nitrate assimilation pathways to uphold redox balance and optimize C-N metabolism under varying N conditions \citep{Gandin2014,Vishwakarma2015}.


\subsection*{Beta‑Sitosterol GWAS Links Cellulose Synthase to Membrane Stability}

In our GWAS of beta-sitosterol (BS), we detected a significant association peak at the cellulose synthase locus SORBI\_3003G049600, indicating that variations in this CesA gene may affect BS accumulation or its function in stabilizing membranes in sorghum. BS is a common phytosterol in plants that integrates into lipid bilayers to regulate membrane fluidity and stability. Although its exact role in cell wall structure is not fully understood, BS is suggested to protect cells from abiotic and biotic stress by enhancing plasma membrane integrity and potentially interacting with cytoplasmic and chloroplast membranes \citep{Sayeed2016}. Studies from Arabidopsis implies that BS plays a role in defense responses, yet a conclusive characterization of its role in cell wall mechanics remains necessary \citep{Sayeed2016}. The cellulose synthase (CesA) complexes are responsible for synthesizing the β-1,4-glucan chains of cellulose, the primary load-bearing polysaccharide in plant cell walls. In Arabidopsis, specific CesA isoforms form plasma-membrane rosettes to produce primary-wall (e.g., AtCesA1, 3, 6) and secondary-wall cellulose (e.g., AtCesA4, 7, 8), which support cell expansion, mechanical strength, and biomass accumulation \citep{Mueller1980,Somerville2006,Hu2018}. Mutations in CesA genes (e.g., \emph{rsw1}, \emph{prc1‑1}) result in decreased cellulose content, weakened cell walls, altered cell morphology, and reduced stress resistance \citep{Hu2018,Arioli1998,Persson2007,CanoDelgado2003,HernandezBlanco2007}. Although CesA primarily directs carbon towards cellulose production, downregulation or mutation of certain CesA genes can redirect carbon flux towards storage compounds. In Arabidopsis seeds, suppression of CesA leads to a slight reduction in cellulose content and prompts compensatory increases in non-cellulosic polysaccharides or proteins \citep{Hu2020}. It has been proposed that redirecting carbon from cell wall polysaccharides to seed storage proteins and oils may enhance nutritional quality, addressing the inverse relationship between seed oil and protein content \citep{Tomlinson2004,Ekman2008,Iyer2008,Shi2012,Tan2011,YoshieStark2008,Knowles1983}. 


\subsection*{Gibberellic Acid Response GWAS Identifies a MADS‑Box Regulator of Flowering Time}

In the gibberellic acid (GA) genome-wide association study (GWAS), we detected the \textit{SORBI\_3007G090421}. GA\(_3\) enhances floral initiation in short-day sorghum genotypes, predominantly when in conjunction with far-red light (FR). Williams and Morgan (1979) demonstrated that the combination of GA\(_3\) and FR results in an advancement of flowering by 30 to 80 days in early to intermediate maturity lines, and independently facilitates stem elongation \citep{Williams1979}. Lee \emph{et al.} (1998) further elucidated that photoperiod and phytochrome B are instrumental in regulating endogenous GA\(_1\)/GA\(_{20}\) rhythms, with altered GA peaks in \emph{phyB}-deficient genotypes being associated with early flowering under non-inductive day lengths \citep{Lee1998}. In our GA\(_3\) GWAS, the MADS-box transcription factor gene SORBI\_3007G090421 was identified. MADS-box proteins, particularly Type II C-function genes, are key regulators of floral organ identity and flowering time, whereas Type I MADS (e.g., \emph{AGL62}) affects endosperm development with consequential indirect effects on reproductive timing \citep{Paul2020}. Environmental temperature influences the effects of GA3 on development; Jabir and Mahmoud (2021) reported that elevated temperatures at planting dates, coupled with GA\(_3\) (100 ppm), expedited sorghum flowering, improved germination, and enhanced enzymatic activities for nutrient mobilization \citep{Jabir2021}. Williams and Morgan also observed genotype-specific temperature responses under controlled versus field conditions, casting light on temperature as a crucial element in GA3-mediated flowering regulation.


\subsection*{GWAS of Zeaxanthin Reveals CP29 as a Key Photoprotective Regulator}

Our GWAS investigation into zeaxanthin identified the CP29-encoding gene SORBI\_3001G357200, suggesting that natural variation in the sequence or expression of CP29 is a key factor in zeaxanthin-mediated photoprotection in sorghum. Zeaxanthin is a carotenoid involved in the xanthophyll cycle, accumulating in thylakoid membranes under excessive light conditions and playing a crucial role in photoprotection. During lumen acidification induced by high light, violaxanthin is converted to zeaxanthin by violaxanthin de-epoxidase, which subsequently associates with specific light-harvesting complex (Lhc) subunits and quenches chlorophyll triplet states, thereby hindering singlet oxygen (\(^{1}\mathrm{O}_{2}\)) formation \citep{DallOsto2012}. Zeaxanthin causes a red shift in carotenoid triplet absorption and facilitates triplet energy transfer from chlorophyll to carotenoid, enhancing non-photochemical quenching (NPQ) and mitigating photoinhibition \citep{DallOsto2012,DemmigAdams2020}. CP29 is a monomeric photosystem II antenna protein that binds three xanthophylls (lutein at L1, violaxanthin/zeaxanthin at L2) and several chlorophylls. The exchange of zeaxanthin at the L2 site induces a charge-transfer quenching state in CP29, which comprises both the fast (qE) and slow components of NPQ \citep{Guardini2020}. Contrary to PsbS, the protonatable residues exposed in the lumen of CP29 are not necessary for the activation of qE; instead, PsbS detects pH changes and interacts with CP29 to promote the conformational change required for energy dissipation \citep{Guardini2020}.



\begin{figure}[htbp]
  \centering
  \includegraphics[width=\textwidth]{fig/main/Fig3.png}
  \caption{\textbf{Low‐input lipid GWAS Manhattan plots for traditional and non‐traditional lipids.}
    \textbf{(A)} Manhattan plots for five TG species (TG\_58\_3, TG\_58\_4, TG\_58\_5, TG\_60\_4, TG\_60\_5), all peaking at an acyl‐CoA:diacylglycerol acyltransferase locus (\textit{SORBI\_3010G170000}) in chromosome 10; the vertical green dashed line marks the SNP position for this gene.  
    \textbf{(B)} SQDG(32:0) Manhattan plot, highlighting a sulfolipid synthase gene (\textit{SORBI\_3002G000600}).  
    \textbf{(C)} \(\alpha\)‐Carotene Manhattan plot, identifying an alternative oxidase (\textit{SORBI\_3006G202500}).  
    \textbf{(D)} \(\beta\)‐Sitosterol Manhattan plot, pinpointing a cellulose synthase (\textit{SORBI\_3003G049600}).  
    \textbf{(E)} Gibberellic acid Manhattan plot, at a MADS‐box transcription factor locus (\textit{SORBI\_3007G090421}).  
    \textbf{(F)} Zeaxanthin Manhattan plot, marking a chloroplast RNA‐binding protein (\textit{SORBI\_3001G357200}).  
    Green dots indicate SNPs within or near the highlighted genes in panels B–F. Panels A and B show traditional lipids (–log\textsubscript{10} $p$\(\geq\)7), and panels C–F show non‐traditional lipids (–log\textsubscript{10} $p$\(\geq\)5).}
  \label{fig:Fig3}
\end{figure}


\subsection*{Lipid Changes under LowInput (LI)}
The multi-year, multi-field experimental design introduced considerable environmental variability into the lipidomic profiles. Although spatial corrections (SpATS) and batch-effect normalization (SERRF) were employed, residual confounding factors necessitated a ratio-based analytical approach to compare the C and the LI conditions. Lipid class ratios, which demonstrate greater resilience to technical and environmental noise than absolute abundances, were prioritized to identify biologically conserved patterns of membrane adaptation. To mitigate these confounding effects, we concentrated on within-plant lipid class ratios, expected to be fairly consistent under a variety of cultivation environments, rather than absolute abundances. We aggregated individual molecular species into their respective lipid classes, computed all possible pairwise ratios on a linear scale, and subsequently applied OPLS-DA to identify the most discriminative ratios. The resulting scores plot (Fig. \ref{fig:Fig2:OPLS}) demonstrated a robust separation of C versus LI samples, and the model exhibited exemplary performance metrics (R²Y = 0.99; Q²Y = 0.82) alongside a highly significant CV-ANOVA (p < 0.001) (Supp Figure, supp table). Hotelling’s T² analysis confirmed the absence of extreme outliers, and a 1,000-permutation test dismissed overfitting, indicating that our observations are unlikely attributable to chance. From the complete ratio set, 21 features surpassed a conservative VIP threshold of 1. Each of these high VIP ratios was then subjected to univariate testing (adjusted p < 0.05) and cross-verified against established pathways of membrane remodeling under cold and nutrient stress. The two-tier selection approach prioritizing statistical power and biological relevance yielded a defined set of lipid-class ratios (supp table) which may differentiate between C and LI. These lipid ratios are used for the further studying the effects of C and LI. 




\begin{figure}[htbp]
  \centering
  \includegraphics[width=\textwidth]{fig/main/Fig2.png}
  \caption{Orthogonal projections to latent structures discriminant analysis (OPLS‐DA) results.  
    \textbf{(A)} Score plot showing Control (purple) vs LowInput (yellow) samples on the predictive component $t_1$ (51.4\% of $X$‐variance) versus the orthogonal component $\mathrm{to}_1$, with 95\% confidence ellipses and dashed axes at zero.  
    \textbf{(B)} Permutation test: overlaid histograms of 200 permuted R²Y (yellow) and Q² (purple) values, with the true‐model metrics indicated by dashed lines (both $p=0.005$), and an arrow marking the cumulative $X$‐variance explained (R²X = 0.514).}
  \label{fig:Fig2:OPLS}
\end{figure}



\begin{table}[!h]
\centering
\caption{\bf Most important lipid ratios based on OLSA-DA}
  \label{table:olsa}

\begin{tabular}{>{}l>{\raggedleft\arraybackslash}p{3cm}}
\toprule
\cellcolor[HTML]{f7f7f7}{\textcolor{black}{\textbf{Lipid}}} & \cellcolor[HTML]{f7f7f7}{\textcolor{black}{\textbf{VIP}}}\\
\midrule
\cellcolor[HTML]{f7f7f7}{\textcolor{black}{\textbf{\cellcolor{gray!10}{MG/SQDG}}}} & \cellcolor[HTML]{f7f7f7}{\textcolor{black}{\cellcolor{gray!10}{1.34}}}\\
\cellcolor[HTML]{f7f7f7}{\textcolor{black}{\textbf{PS/SQDG}}} & \cellcolor[HTML]{f7f7f7}{\textcolor{black}{1.32}}\\
\cellcolor[HTML]{f7f7f7}{\textcolor{black}{\textbf{\cellcolor{gray!10}{MG/MGDG}}}} & \cellcolor[HTML]{f7f7f7}{\textcolor{black}{\cellcolor{gray!10}{1.31}}}\\
\cellcolor[HTML]{f7f7f7}{\textcolor{black}{\textbf{PG/SQDG}}} & \cellcolor[HTML]{f7f7f7}{\textcolor{black}{1.31}}\\
\cellcolor[HTML]{f7f7f7}{\textcolor{black}{\textbf{\cellcolor{gray!10}{DG/MG}}}} & \cellcolor[HTML]{f7f7f7}{\textcolor{black}{\cellcolor{gray!10}{1.30}}}\\
\cellcolor[HTML]{f7f7f7}{\textcolor{black}{\textbf{MGDG/PS}}} & \cellcolor[HTML]{f7f7f7}{\textcolor{black}{1.30}}\\
\cellcolor[HTML]{f7f7f7}{\textcolor{black}{\textbf{\cellcolor{gray!10}{DGDG/MG}}}} & \cellcolor[HTML]{f7f7f7}{\textcolor{black}{\cellcolor{gray!10}{1.29}}}\\
\cellcolor[HTML]{f7f7f7}{\textcolor{black}{\textbf{LPC/MG}}} & \cellcolor[HTML]{f7f7f7}{\textcolor{black}{1.29}}\\
\cellcolor[HTML]{f7f7f7}{\textcolor{black}{\textbf{\cellcolor{gray!10}{LPC/PS}}}} & \cellcolor[HTML]{f7f7f7}{\textcolor{black}{\cellcolor{gray!10}{1.29}}}\\
\cellcolor[HTML]{f7f7f7}{\textcolor{black}{\textbf{PE/SQDG}}} & \cellcolor[HTML]{f7f7f7}{\textcolor{black}{1.29}}\\
\cellcolor[HTML]{f7f7f7}{\textcolor{black}{\textbf{\cellcolor{gray!10}{DG/PS}}}} & \cellcolor[HTML]{f7f7f7}{\textcolor{black}{\cellcolor{gray!10}{1.29}}}\\
\cellcolor[HTML]{f7f7f7}{\textcolor{black}{\textbf{LPE/SQDG}}} & \cellcolor[HTML]{f7f7f7}{\textcolor{black}{1.28}}\\
\cellcolor[HTML]{f7f7f7}{\textcolor{black}{\textbf{\cellcolor{gray!10}{DGDG/PS}}}} & \cellcolor[HTML]{f7f7f7}{\textcolor{black}{\cellcolor{gray!10}{1.28}}}\\
\cellcolor[HTML]{f7f7f7}{\textcolor{black}{\textbf{PC/PS}}} & \cellcolor[HTML]{f7f7f7}{\textcolor{black}{1.27}}\\
\cellcolor[HTML]{f7f7f7}{\textcolor{black}{\textbf{\cellcolor{gray!10}{DGDG/SQDG}}}} & \cellcolor[HTML]{f7f7f7}{\textcolor{black}{\cellcolor{gray!10}{1.24}}}\\
\cellcolor[HTML]{f7f7f7}{\textcolor{black}{\textbf{LPC/LPE}}} & \cellcolor[HTML]{f7f7f7}{\textcolor{black}{1.24}}\\
\cellcolor[HTML]{f7f7f7}{\textcolor{black}{\textbf{\cellcolor{gray!10}{PE/PS}}}} & \cellcolor[HTML]{f7f7f7}{\textcolor{black}{\cellcolor{gray!10}{1.2}}}\\
\cellcolor[HTML]{f7f7f7}{\textcolor{black}{\textbf{PG/PS}}} & \cellcolor[HTML]{f7f7f7}{\textcolor{black}{1.23}}\\
\cellcolor[HTML]{f7f7f7}{\textcolor{black}{\textbf{\cellcolor{gray!10}{SQDG/TG}}}} & \cellcolor[HTML]{f7f7f7}{\textcolor{black}{\cellcolor{gray!10}{1.23}}}\\
\cellcolor[HTML]{f7f7f7}{\textcolor{black}{\textbf{MGDG/PG}}} & \cellcolor[HTML]{f7f7f7}{\textcolor{black}{1.22}}\\
\cellcolor[HTML]{f7f7f7}{\textcolor{black}{\textbf{\cellcolor{gray!10}{MG/PG}}}} & \cellcolor[HTML]{f7f7f7}{\textcolor{black}{\cellcolor{gray!10}{1.22}}}\\
\cellcolor[HTML]{f7f7f7}{\textcolor{black}{\textbf{PA/PS}}} & \cellcolor[HTML]{f7f7f7}{\textcolor{black}{1.21}}\\
\cellcolor[HTML]{f7f7f7}{\textcolor{black}{\textbf{\cellcolor{gray!10}{MG/PC}}}} & \cellcolor[HTML]{f7f7f7}{\textcolor{black}{\cellcolor{gray!10}{1.20}}}\\
\cellcolor[HTML]{f7f7f7}{\textcolor{black}{\textbf{MGDG/PE}}} & \cellcolor[HTML]{f7f7f7}{\textcolor{black}{1.19}}\\
\cellcolor[HTML]{f7f7f7}{\textcolor{black}{\textbf{\cellcolor{gray!10}{PC/SQDG}}}} & \cellcolor[HTML]{f7f7f7}{\textcolor{black}{\cellcolor{gray!10}{1.17}}}\\
\cellcolor[HTML]{f7f7f7}{\textcolor{black}{\textbf{LPE/MGDG}}} & \cellcolor[HTML]{f7f7f7}{\textcolor{black}{1.16}}\\
\cellcolor[HTML]{f7f7f7}{\textcolor{black}{\textbf{\cellcolor{gray!10}{DGDG/MGDG}}}} & \cellcolor[HTML]{f7f7f7}{\textcolor{black}{\cellcolor{gray!10}{1.16}}}\\
\cellcolor[HTML]{f7f7f7}{\textcolor{black}{\textbf{MG/PA}}} & \cellcolor[HTML]{f7f7f7}{\textcolor{black}{1.16}}\\
\cellcolor[HTML]{f7f7f7}{\textcolor{black}{\textbf{\cellcolor{gray!10}{LPE/PS}}}} & \cellcolor[HTML]{f7f7f7}{\textcolor{black}{\cellcolor{gray!10}{1.15}}}\\
\cellcolor[HTML]{f7f7f7}{\textcolor{black}{\textbf{PS/TG}}} & \cellcolor[HTML]{f7f7f7}{\textcolor{black}{1.14}}\\
\cellcolor[HTML]{f7f7f7}{\textcolor{black}{\textbf{\cellcolor{gray!10}{MG/PE}}}} & \cellcolor[HTML]{f7f7f7}{\textcolor{black}{\cellcolor{gray!10}{1.13}}}\\
\cellcolor[HTML]{f7f7f7}{\textcolor{black}{\textbf{MGDG/TG}}} & \cellcolor[HTML]{f7f7f7}{\textcolor{black}{1.13}}\\
\cellcolor[HTML]{f7f7f7}{\textcolor{black}{\textbf{\cellcolor{gray!10}{DG/LPE}}}} & \cellcolor[HTML]{f7f7f7}{\textcolor{black}{\cellcolor{gray!10}{1.1}}}\\
\cellcolor[HTML]{f7f7f7}{\textcolor{black}{\textbf{LPC/PE}}} & \cellcolor[HTML]{f7f7f7}{\textcolor{black}{1.08}}\\
\cellcolor[HTML]{f7f7f7}{\textcolor{black}{\textbf{\cellcolor{gray!10}{LPE/PA}}}} & \cellcolor[HTML]{f7f7f7}{\textcolor{black}{\cellcolor{gray!10}{1.08}}}\\
\cellcolor[HTML]{f7f7f7}{\textcolor{black}{\textbf{DG/TG}}} & \cellcolor[HTML]{f7f7f7}{\textcolor{black}{1.07}}\\
\cellcolor[HTML]{f7f7f7}{\textcolor{black}{\textbf{\cellcolor{gray!10}{PA/TG}}}} & \cellcolor[HTML]{f7f7f7}{\textcolor{black}{\cellcolor{gray!10}{1.06}}}\\
\cellcolor[HTML]{f7f7f7}{\textcolor{black}{\textbf{LPC/PG}}} & \cellcolor[HTML]{f7f7f7}{\textcolor{black}{1.06}}\\
\cellcolor[HTML]{f7f7f7}{\textcolor{black}{\textbf{\cellcolor{gray!10}{DG/SQDG}}}} & \cellcolor[HTML]{f7f7f7}{\textcolor{black}{\cellcolor{gray!10}{1.05}}}\\
\cellcolor[HTML]{f7f7f7}{\textcolor{black}{\textbf{MGDG/PC}}} & \cellcolor[HTML]{f7f7f7}{\textcolor{black}{1.04}}\\
\cellcolor[HTML]{f7f7f7}{\textcolor{black}{\textbf{\cellcolor{gray!10}{DG/PG}}}} & \cellcolor[HTML]{f7f7f7}{\textcolor{black}{\cellcolor{gray!10}{1.03}}}\\
\cellcolor[HTML]{f7f7f7}{\textcolor{black}{\textbf{LPC/TG}}} & \cellcolor[HTML]{f7f7f7}{\textcolor{black}{1.02}}\\
\textcolor{black}{\cellcolor[HTML]{f7f7f7}{\textbf{\textbf{\cellcolor{gray!10}{DG/PE}}}}} & \textcolor{black}{\cellcolor[HTML]{f7f7f7}{{\cellcolor{gray!10}{1}}}}\\
\bottomrule
\end{tabular}
\end{table}

Important ratios 


SORBI_3006G214500 - Lecithin cholesterol acyltransferase like 4
SORBI_3001G448800 - Lecithin cholesterol acyltransferase like 1
SORBI_3001G041900 - PNLPA
SORBI_3001G042901 - Adenylate isopentanyltransferases


\subsection*{1. Membrane Lipid Remodeling under LowInput (LI)}

\subsubsection*{1.1 Sulfolipid (SQDG) Collapse Under LI Stress Adaptation}

In plants, N uptake commonly occurs mainly in the form of nitrate (NO₃⁻), which is subsequently reduced by the enzyme nitrate reductase.
Simultaneously, we have developed the BZea population as a broadly useful teosinte–introgression resource. Teosinte, the wild ancestor of maize, possesses advantageous traits for nitrogen dynamics that were may have been reduced during the domestication process. Our collection consists of approximately 2,100 BC₂S₃ lines derived from 81 georeferenced teosinte accessions encompassing Zea mays ssp. parviglumis, ssp. huehuetenangensis, ssp. mexicana, Z. diploperennis, and Z. luxurians. Through backcrossing with B73, each line conserves approximately 12.5\% of the teosinte genetic material while remaining agronomically viable under temperate conditions, thereby overcoming photoperiod and growth discrepancies. We have conducted whole-genome sequencing on 1,400 lines at approximately 0.8× coverage and generated 3′ RNA-seq profiles for population characterization. Current efforts are focused on examining nitrogen dynamics in lines bearing Z. diploperennis alleles, which are hypothesized to be enhanced for nutrient recycling. More broadly, we are evaluating loci nominated by environmental genome-wide association studies by selecting BZea introgression lines that contain the pertinent teosinte haplotypes. The integration of these strategies seeks to elucidate the functional contributions of wild alleles to the nitrogen biology of maize.


Teosinte, the wild ancestor progenitor of maize, contains an array of advantageous traits, notably Nitrogen Use Efficiency (NUE), which have been substantially reduced through considerably diminished during the domestication of maize. Here we provide both the This report provides a comprehensive genotypic and phenotypic characterization of a newly developed teosinte introgression population, along with preliminary observations pertinent to the study of NUE in this population. The donor lines of the population consist of 81 georeferenced teosinte accessions, including multiple species within the Zea genus, such as Zea mays ssp. parviglumis, ssp. huehuetenangensis, ssp. mexicana, Zea diploperennis, and Zea luxurians, resulting in approximately 2100 derived lines. The photoperiod and growth requisites of teosinte present challenges for the examination of NUE. The BZea panel addresses this challenge by utilizing BC₂S₃ lines that have undergone backcrossing to B73, thus containing approximately 12.5% teosinte DNA in each line. Whole-genome sequencing was conducted on 1,400 BZea BC₂S₃ lines at an approximate coverage of 0.8×, alongside 3’ RNA-sequencing for population characterization. To investigate NUE in maize, an environmental genome-wide analysis was performed on a georeferenced panel of landraces, which exhibit adaptation across the Americas, in relation to total soil nitrogen. This led to the identification of several candidates for nitrogen-related phenotypes, including a TATC Sec-independent transporter and genes involved in the aspartate metabolism pathway. A CRISPR-knockout of the TATC Sec-independent gene has been produced and is presently undergoing phenotypic analysis to assess its impact on NUE. Concurrently, a comprehensive evaluation of nitrogen dynamics is being conducted in lines incorporating alleles from Zea diploperennis, a perennial teosinte potentially enriched with nutrient-recycling capabilities. Furthermore, functional testing of these candidates is underway within the BZea panel by selecting introgression lines that harbor teosinte haplotypes at the relevant loci. These initiatives aim to elucidate the functional significance of teosinte alleles and their potential to improve agronomic traits in maize.

Teosinte, the wild ancestor of maize, harbors an abundance of valuable traits that have been substantially reduced through the domestication process of maize. Recognizing the potential of these unexploited alleles, we provide both the genotypic and phenotypic characterization of a newly developed teosinte introgression population, along with preliminary insights into the experimental applicability of this genetic material. The donor lines of the population include 81 georeferenced teosinte accessions encompassing multiple species within the Zea genus, such as Zea mays ssp. parviglumis, ssp. huehuetenangensis, ssp. mexicana, Zea diploperennis, and Zea luxurians, culminating in approximately 2100 derived lines. Evaluating the impact of teosinte alleles on agronomic traits presents a challenge due to the photoperiod and growth environment differences between maize and teosinte. The 'BZea' population addresses this complexity by employing BC2S3s with B73, resulting in derived lines that retain 12.5\% of the original teosinte genetic composition. This introgression into B73 enables the assessment of teosinte alleles within a maize framework under temperate conditions. To elucidate the characteristics of this population, we performed whole genome sequencing on 1400 of these derived lines with an average sequencing depth of 0.8x. Additionally, we executed 3’ RNA-sequencing on a subset of the population. Through this population, we are exploring various experimental avenues. Currently, we are examining nitrogen dynamics employing lines from Z. diploperennis, a perennial teosinte presumed to possess alleles relevant to nutrient recycling. Furthermore, we utilize this population to construct allelic series for numerous candidate genes, including those related to nitrogen and other environmental factors. These efforts aim to uncover the functional roles of teosinte alleles and their potential to enhance maize agronomic traits.

Nitrogen Use Efficiency (NUE) is a critical factor in augmenting crop productivity, as it quantifies the efficacy with which plants utilize applied nitrogen (N) fertilizers. It is typically defined as the grain yield produced per unit of nitrogen supplied. Enhancing NUE is essential for attaining sustainable agriculture due to its potential to diminish reliance on N fertilizers and mitigate environmental damage. Aspartate serves as a fundamental metabolite in nitrogen metabolism, acting as a precursor for the biosynthesis of key amino acids such as lysine, threonine, and methionine, and is crucial for N assimilation. Modifications in aspartate metabolism can have a direct impact on NUE by improving nitrogen remobilization under low-N conditions. Natural genetic variability exists within crop populations, offering opportunities to enhance N metabolism and NUE. Natural allelic diversity in genes vital for N uptake and assimilation has been shown to advance NUE across various species. In sorghum, genetic variation in traits related to NUE has been documented, with specific alleles resulting in enhanced nitrogen uptake and metabolic efficiency. In our study, we employed Genome-Wide Association Studies (GWAS) and Fst analysis to identify key genetic loci associated with NUE in maize. Our findings highlighted the significance of the aspartate metabolism pathway in nitrogen assimilation and redistribution. Through GWAS, we identified a candidate gene with a strong association with N-related traits, suggesting its potential role in regulating nitrogen metabolism. Furthermore, Fst analysis of the Indian Chief and Jarvis maize populations, which have undergone extensive selection (14 cycles) for increased yield and N accumulation in seeds, revealed distinct selection signals at various stages in the aspartate pathway. These results imply that modifying aspartate metabolism could enhance NUE under selective pressure. Future endeavors will concentrate on the functional validation of these candidate genes or entire pathways through CRISPR-based gene knockouts and overexpression studies. Additionally, several other metabolic pathways under selection were identified, although they are not examined here in detail. These pathways might collaborate in supportive roles related to nitrogen uptake, transport, assimilation, and remobilization. It is crucial to consider these pathways when developing new high-NUE maize varieties.


% ------------------------------------------------------------------
%  Canonical behaviour of SQDG under single stresses
% ------------------------------------------------------------------
\begin{table}[!ht]
  \centering
  \caption{\bf Canonical responses of sulfoquinovosyldiacylglycerol (SQDG) under single abiotic stresses.}
  \label{table:SQDG_responses}
  \begin{tabular}{@{} l l c p{6cm} c @{}}
    \toprule
    \textbf{Class} 
      & \textbf{Stress} 
      & \textbf{Response} 
      & \textbf{Mechanism} 
      & \textbf{Ref.} \\
    \midrule
    \multirow{3}{*}{\textbf{SQDG}}
      & Low-P  & ↑ (strong induction)   
               & SQDG replaces anionic phospholipids (mainly PG) to maintain thylakoid surface charge while economizing phosphorus; up-regulation of SQD1/SQD2 diverts UDP-sulfoquinovose toward SQDG synthesis. 
               & [1] \\
      & Low-N  & ↓/≈ (variable)         
               & Nitrogen scarcity limits sulfate assimilation and UDP-sulfoquinovose production; DAG backbones are redirected to TAG or DGDG rather than SQDG. 
               & [2] \\
      & Cold   & ↑/≈ (species-dependent)
               & Cold acclimation often elevates SQDG for extra anionic buffering of PSII complexes; responses vary with species and sugar availability. 
               & [3] \\
    \bottomrule
  \end{tabular}
  \begin{flushleft}
    {\footnotesize Arrows denote relative change vs.\ control (↑ increase, ↓ decrease, ≈ no consistent change).}
  \end{flushleft}
\end{table}



\paragraph{Key Observations}
\begin{enumerate}
  \item \textbf{Preferential SQDG depletion relative to phospholipids: } \\ 
  Both SQDG/PC and SQDG/PG decrease significantly (LI $<$ Control, $p<0.001$), indicating that SQDG deteriorates more rapidly compared to the primary phospholipids in extraplastidial (\textit{PC}) and thylakoid (\textit{PG}) membranes.
  
  \item \textbf{SQDG loss outpaces DGDG: }\\
  SQDG decreases more significantly than DGDG ($p<0.001$), suggesting that DGDG is conserved to maintain bilayer stability. The potential high metabolic costs in terms of sulfur and energy, as well as its non-essential nature for short-term membrane integrity, make the breakdown of SQDG more prioritized. We define {SQDG\textsubscript{total}} as \(\bigl[\mathrm{SQDG}-\tfrac12(\mathrm{DGDG}+\mathrm{MGDG})\bigr]\). Furthermore, the total SQDG levels decline in LI ($p<0.001$), indicating an overall sulfolipid deficit. However, MGDG levels remain unchanged with respect to SQDG.
  
\end{enumerate}

\paragraph{Metabolic Interpretation and Physiological Significance}
\vspace{-0.5ex}
\begin{enumerate}
  \item \textbf{Sulfur salvage for stress defence.} \\ 
        SQDG breakdown yields sulfoquinovose, contributing sulfur for glutathione and other thiols that neutralize ROS during the initial phase of simultaneous cold and nutrient scarcity.
 % \item \textit{DG liberation fuels downstream pathways.}  
 %       DAG produced from SQDG hydrolysis explains the sharp rise in
 %       \(\mathrm{DG/SQDG}\) (Section 2); that DAG is later diverted to
 %       TAG rather than recycled into membranes.
  \item \textbf{Anionic‐lipid charge compensation.} \\
        The loss of SQDG is typically compensated by an increase in PG or DGDG during phosphorus deficiency. In this scenario, both the SQDG/Phospho and SQDG/Galacto ratios decrease, indicating that LI plants may tolerate a net decrease in anionic surface charge, which might be stabilized by the association with Ca\(^{2+}\) or Mg\(^{2+}\) in cold conditions.
\end{enumerate}



\subsubsection*{1.2 Galactolipid Dynamics}

 In LI treatment, galactolipid degradation becomes asymmetrical, with MGDG experiencing a greater depletion relative to DGDG. This imbalance signifies a tactical balance between energy efficiency and maintaining membrane integrity (see Fig.~\ref{fig:lipid_remodeling}, “Galactolipid Dynamics”):
% ------------------------------------------------------------------
%  Galactolipid behaviour (DGDG and MGDG)
% ------------------------------------------------------------------
\begin{table}[!ht]
  \centering
  \caption{\bf Canonical responses of the two major galactolipids (DGDG and MGDG) under single abiotic stresses.}
  \label{table:galactolipid_responses}
  \begin{tabularx}{\textwidth}{@{} l l p{3.5cm} X c @{}}
    \toprule
    \textbf{Class}
      & \textbf{Stress}
      & \textbf{Response}
      & \textbf{Mechanism}
      & \textbf{Ref.} \\
    \midrule
    \multirow{3}{*}{\textbf{Galactolipids}}
      & Low-P  
         & \begin{minipage}[t]{3.5cm}
             DGDG: ↑ replaces PC/PE;\\
             MGDG: ↑ replaces PC/PE
           \end{minipage}
         & DAG is redirected toward galactolipid synthesis, substituting for phospholipids under P limitation.
         & [1] \\[1ex]
      & Low-N  
         & \begin{minipage}[t]{3.5cm}
             DGDG: ↑/≈ (weak retention);\\
             MGDG: ↓
           \end{minipage}
         & Enhanced PC breakdown releases DAG that is partly channeled to DGDG, while MGDG declines as resources shift to storage lipids or other pathways.
         & [2] \\[1ex]
      & Cold   
         & \begin{minipage}[t]{3.5cm}
             DGDG: ↑ (bilayer stabilizer);\\
             MGDG: ↓ (converted or remodeled)
           \end{minipage}
         & Cold stress favors higher DGDG for maintaining bilayer stability; MGDG is remodeled or reduced to adjust membrane fluidity.
         & [3] \\
    \bottomrule
  \end{tabularx}
  \begin{flushleft}
    {\footnotesize Arrows denote relative changes vs.\ control (↑ increase, ↓ decrease, ≈ no significant change).}
  \end{flushleft}
\end{table}

\paragraph{Key Observations}
\begin{enumerate}
  \item \textbf{MGDG is the Primary Target of Degradation} \\
  The ratio \(\mathrm{MGDG/DGDG}\) decreases significantly (\(p<0.001\)), highlighting a more rapid depletion of MGDG compared to its bilayer-supporting counterpart, DGDG. This pattern may imply that stress-activated lipases, such as PLA1, preferentially hydrolyze MGDG. Additionally, the \(\mathrm{MGDG/PC}\) ratio and the composite measure \(\mathrm{MGDG}_{\mathrm{total}} = \mathrm{MGDG} - \tfrac12(\mathrm{DGDG}+\mathrm{SQDG})\) both show substantial reductions (\(p<0.001\)), indicating that MGDG is the primary lipid reserve to be depleted under LI stresses.
  
  \item \textbf{DGDG is Relatively Preserved but Outcompeted by Phospholipids} \\
  DGDG\_total which is \([\mathrm{DGDG} - \tfrac12(\mathrm{MGDG}+\mathrm{SQDG})]\) shows a significant increase (\(p<0.001\)), suggesting relative preservation (Fig. TIC). This retention of DGDG implies that membrane collapse is being avoided. Furthermore, both \(\mathrm{DGDG/PC}\) and \(\mathrm{DGDG/PE}\) decrease (\(p<0.001\)), indicating that while DGDG levels surpass those of MGDG, they are still being overtaken by phospholipid concentrations under stress. Interestingly, \(\mathrm{DGDG/PG}\) exhibits an increase (\(p<0.001\)), which indicates a shift in DGDG distribution that might stabilize photosystem II (PSII) complexes. The increase in PG$_{\text{retention}}$ underscores PG's active preservation relative to other phospholipids like PC and PE, aligning with its vital function in photosynthesis.
\end{enumerate}

\paragraph{Potential Physiological Implications}
\begin{enumerate}
  \item \textbf{Thylakoid Membrane Rigidification via Lipid Restructuring} \\
        The reduction in MGDG/DGDG suggests a shift from the non-bilayer arrangement of MGDG to the stable bilayer form of DGDG. This change likely serves to counteract fluidity alterations caused by cold and nutrient stress, thus increasing membrane rigidity. Furthermore, it could aid in preventing ion leakage during nutrient shortages, given that the bilayer structure of DGDG provides superior seal integrity compared to the hexagonal II phase inclination of MGDG.
  
  \item \textbf{Carbon channeling within galactolipid pathways.} \\ 
        The simultaneous reduction of MGDG\_total while total DGDG\_total levels are maintained or increased indicates a prioritization in resource allocation, where the breakdown of MGDG supports the continued synthesis of DGDG during LI stress conditions. This suggests that the activity of DGD1/2 synthase could remain active for a more extended period or resist low temperatures more effectively than MGD1/3 activities, possibly due to differences in enzyme stability and the requirement for a bilayer during stress. The degradation of MGDG may release carbon, which is then used for the ongoing synthesis of residual DGDG, mediated through recycled DG, and for the storage of TG (refer to the increase in TG/DG in Section 1.3).
  
  \item \textbf{Dual functionality of DGDG} \\ 
        The rise in DGDG/PG indicates that DGDG may partially substitute for PG’s anionic surface charge. %By shuffling anionic lipids—DGDG and PG—to the forefront, LI plants may also optimize binding sites for Ca\(^{2+}\) and Mg\(^{2+}\) ions. These cations bridge adjacent negatively charged headgroups, tightening bilayer packing and mitigating leakiness that would otherwise arise from headgroup depletion.
        PG is limited (↓ PG\_retention, \(p<0.001\)), and SQDG is inadequate in this capacity (↓ SQDG/PG, \(p<0.001\)), which underscores the unique dual role of DGDG as both a structural lipid and a charge buffer.
\end{enumerate}

%\paragraph{Proposed Regulatory Crossroads}
%\begin{itemize}
%  \item \textit{Differential enzyme sensitivities:}  
%        MGDG synthase (MGD1/2) activity may be more inhibited by combined Low-N/P/cold than DGDG synthases, leading to sharper MGDG loss.
%  \item \textit{Galactose precursor limitation:}  
%        Reduced UDP-galactose under nutrient/cold stress may throttle MGDG production more than DGDG, whose existing pools are preferentially maintained.
%\end{itemize}



\subsubsection*{1.3 Phospholipid Homeostasis}

Under LI stress, phospholipid pools are selectively reorganized (Fig.~\ref{fig:lipid_remodeling} “Phospholipid Homeostasis”):

\begin{table}[!ht]
  \centering
  \caption{\bf Canonical responses of PC, LPC and PG under single abiotic stresses.}
  \label{table:lipid_responses}
  \begin{tabular}{@{} l l c p{5.5cm} c @{}}
    \toprule
    \textbf{Class} 
      & \textbf{Stress} 
      & \textbf{Response} 
      & \textbf{Mechanism} 
      & \textbf{Ref.} \\
    \midrule
    \multirow{3}{*}{\textbf{PC}} 
      & Low-P  & ↓ (major P donor)      
               & Phospholipase-mediated PC hydrolysis releases choline and phosphate; DAG backbone is redirected to DGDG or TAG synthesis. 
               & [1] \\
      & Low-N  & ↓ (N-source)          
               & Choline head-group catabolism supplies nitrogen; DAG is channeled into TAG accumulation. 
               & [2] \\
      & Cold   & ↑/≈ (species‐dependent) 
               & Increased PC unsaturation preserves bilayer fluidity; net PC often maintained or slightly elevated. 
               & [3] \\
    \addlinespace
    \multirow{3}{*}{\textbf{LPC}} 
      & Low-P  & ↑ (lipase product)    
               & PLA$_2$ activity rises to liberate phosphate; LPC is subsequently exported or re-acylated. 
               & [4] \\
      & Low-N  & ↓/≈                   
               & Reduced phospholipase activity conserves N-rich head-groups; DAG is preferred for storage rather than LPC re-acylation. 
               & [5] \\
      & Cold   & ↑                      
               & Rapid acyl-editing cycle (LPC ↔ PC) supports insertion of unsaturated acyl chains for cold acclimation. 
               & [6] \\
    \addlinespace
    \multirow{3}{*}{\textbf{PG}} 
      & Low-P  & ↓/≈                   
               & Partially substituted by SQDG to conserve phosphorus, yet basal PG is retained for PSII function. 
               & [7] \\
      & Low-N  & ≈/↓                   
               & PG contains no nitrogen; levels remain largely stable with only minor decline if DAG is diverted to TAG. 
               & [8] \\
      & Cold   & ↑ (thylakoid stabilizer)
               & Extra PG supports PSII dimer stability and optimizes electron transport at low temperature. 
               & [9] \\
    \bottomrule
  \end{tabular}
  \begin{flushleft}
    {\footnotesize Arrows denote relative change vs.\ control (↑ increase, ↓ decrease, ≈ no consistent change).}
  \end{flushleft}
\end{table}

\begin{enumerate}
  \item \textbf{PC/PE Asymmetry Suggests Membrane Rigidification} \\
  The substantial rise in PC/PE (\(p<0.001\)) suggests a favored preservation or production of PCs rather than PEs. The methylated headgroup of PC contributes to enhancing bilayer stability under conditions of low temperature and limited nutrients, whereas PE, with its smaller headgroup and tendency toward non-bilayer formations, is comparatively reduced.
  
  \item \textbf{PC/PG Increase Reflects Photosynthetic Trade-offs} \\
  PG retention is specific to plastids (refer to ↑DGDG/PG, section 1.2), though its overall quantity diminishes relative to PC. This reduction might aid in maintaining non-plastidic membrane structure as PG is depleted. Additionally, the increase in PC could potentially counterbalance the loss of thylakoid lipids (as indicated by ↓ MGDG\_retention and ↓ SQDG\_retention, section 1.2, Galactolipids, Dynamics).
  
  \item \textbf{PG\_retention Decline Highlights Metabolic Cost} \\
  The retention of PG (\(\mathrm{PG} - \tfrac12(\mathrm{PC} + \mathrm{PE})\)) decreases significantly (\(p<0.001\)), indicating that although PG is actively preserved for thylakoid functionality, its overall pool diminishes when compared to total phospholipids. The formation of PG depends on CDP-DAG and glycerol-3P, processes that are energy-demanding and may be reallocated under stress conditions.
  
  \item \textbf{NonP\_Phosphorus Decline Signals P-Limitation} \\
  The metric for non-P lipids, represented by \(\bigl[(\mathrm{MGDG}+\mathrm{DGDG})-(\mathrm{PC}+\mathrm{PE})\bigr]\), shows a significant decrease (\(p<0.001\)). This indicates that galactolipids and sulfolipids are less prominent compared to phospholipids, despite being non-phosphorus lipids. This observation deviates from the expected behavior during classic P starvation. In contrast to P limitation alone, the combined stress does not lead to an increase in the DGDG/PC ratio (↓ DGDG/PC), implying that the constraints from cold and nitrogen conditions overshadow the typical lipid shifts associated with P deficiency.

\end{enumerate}

\paragraph{Physiological Implications}
\begin{enumerate}
  \item \textbf{Bilayer fluidity prioritization.}  \\
        Elevated PC/PE ratio indicates that plants treated with LI alter headgroup composition to keep membrane viscosity optimal in cold, low N/P environments.
  
  \item \textbf{Selective anionic lipid allocation.}  \\
        While there is an overall reduction in PG, the rise in PC/PG ratios and the dynamics of PG retention suggest a dual approach: most of the PG is forfeited, yet a crucial portion is conserved within photosynthetic membranes.> 
  
  \item \textbf{Failure of classic galactolipid replacement.}  \\
        The decrease in the non-P:P metric indicates that, unlike straightforward P-starvation, LI plants are unable to replace phospholipids with galacto- and sulfolipids, probably due to simultaneous cold, N and P restrictions.
\end{enumerate}

%\paragraph{Proposed Regulatory Crossroads}
%\begin{itemize}
%  \item \textit{PC/PE Balance Control Point:}  
%        There might be a preferential activation of the Kennedy pathway's PC branch (via CTP:phosphocholine cytidylyltransferase). Similarly a potential PE N-methyltransferase (PEAMT) up regulation. Selective retention of PC through suprressed phospholipase D (PLD) activity. This makes is contrast with the classic P-starvation response. 
%  \item \textit{PG Biosynthesis Bottleneck:}  
%        Declining PG retention despite photosynthetic needs indicates that CDP-DAG:glycerol-3-phosphate phosphatidyltransferase (PGP1) becomes rate-limited. Possible competition for glycerol-3-phosphate (diverted to glycolysis/TAG). There could also be a possible cold-induced enzyme instability in PG synthesis pathway. 
%  \item \textit{Galactolipid Substitution Blockade:}  
%        Failed non-P lipid substitution (NonP\_Phosphorus) reveals interesting results. Due to N-limitation, UDP-galactose pool might be depleted which are required for the galactolipids. MGD/DGD synthase that produces the galactolipids might be inhibited due to the cold stress as well. We see that the carbon flux is redirected towards the PC/TAG rather than the galactolipids.      
%\end{itemize}


\subsection*{2. Lipid Turnover and Signaling under LowInput (LI)}

LI stress profoundly alters DG dynamics and lysophospholipid metabolism, reflecting a shift from membrane maintenance toward emergency carbon redeployment (Fig.~\ref{fig:lipid_remodeling} “Lipid Turnover and Signaling”):

\subsubsection*{2.1 Diacylglycerol (DG) Flux Hub}
\begin{enumerate}
  \item \textbf{DG Flux Hub Reveals Metabolic Bottlenecks} \\
  The significant rise in both DG/DGDG and DG/MGDG (\(p<0.001\)) suggests a quick buildup or insufficient recycling of DG as galactolipids break down, which implies that DAG kinase activity may be inhibited within plastids.
  
  \item \textbf{Cytosolic DG Starvation} \\
  DG/PC and DG/PE show a notable reduction (\(p<0.001\)), indicating that this DG pool is not reintroduced into the principal phospholipids. In the same vein, DG\_Phospho \(\bigl[\mathrm{DG} - \tfrac12(\mathrm{PC}+\mathrm{PE})\bigr]\) also decreases (\(p<0.001\)). This implies that the Kennedy pathway circumvents free DG and utilizes CDP-DAG directly. It also reveals PLD inhibition (↓lysoPLs), which curtails phospholipid-derived DG.
  
  \item \textbf{SQDG-DG Coupling}
  Rising levels of DG/SQDG (\(p<0.001\)) align with the degradation of SQDG providing DG backbones, which initially accumulate temporarily, before being channeled elsewhere (Section 1.1).
\end{enumerate}

% ------------------------------------------------------------------
%  Canonical behaviour of diacylglycerol (DG) under single stresses
% ------------------------------------------------------------------
\begin{table}[!ht]
  \centering
  \caption{\bf Canonical responses of diacylglycerol (DG) under single abiotic stresses.}
  \label{table:DG_responses}
  \begin{tabularx}{\textwidth}{@{} l l p{3.5cm} X c @{}}
    \toprule
    \textbf{Class}
      & \textbf{Stress}
      & \textbf{Response}
      & \textbf{Mechanism}
      & \textbf{Ref.} \\
    \midrule
    \multirow{3}{*}{\textbf{DG}}
      & Low-P  
         & \begin{minipage}[t]{3.5cm}
             DG: ↑ (phospholipase product)
           \end{minipage}
         & PC/PE hydrolysis liberates DAG backbones that serve as precursors for DGDG or TAG when phosphate is scarce.
         & [1] \\[1ex]
      & Low-N  
         & \begin{minipage}[t]{3.5cm}
             DG: ↑↑ (strong accumulation)
           \end{minipage}
         & Choline head-group catabolism from PC generates surplus DAG; limited acyl-CoA supply slows reconversion, so DAG pools rise.
         & [2] \\[1ex]
      & Cold   
         & \begin{minipage}[t]{3.5cm}
             DG: ↑ / transient pulse
           \end{minipage}
         & Rapid phospholipase A/C activity produces DAG for acyl remodelling; a transient DAG spike precedes re-acylation into cold-adapted PC or TAG.
         & [3] \\
    \bottomrule
  \end{tabularx}
  \begin{flushleft}
    {\footnotesize Arrows denote relative change vs.\ control (↑ increase, ↑↑ strong increase, ↓ decrease, ≈ no consistent change).  Replace reference placeholders with appropriate citations.}
  \end{flushleft}
\end{table}


\paragraph{Physiological Implications}
\begin{enumerate}
  \item \textbf{DG as a transient carbon hub.}  \\
        The increase in DG (DG/DGDG, DG/MGDG, DG/SQDG), along with its absence from phospholipid pools (DG/PC, DG/PE, DG\_phospho), indicates that DG is being redirected towards storage lipids such as TG, emphasizing the formation of energy reserves over the urgent need for membrane repair.
\end{enumerate}


 % \item \textit{Shift toward storage lipid synthesis.}  
 %       Accumulated DG likely feeds into triacylglycerol synthesis (see Section 3), reflecting a strategic rerouting of carbon from membrane maintenance to energy storage during prolonged stress.

\subsubsection*{2.2 Lysophospholipid Remodeling}
\begin{enumerate}
  \item \textbf{Global Lyso-phospholipid Reduction} \\
  LPC/PC, LPE/PE, and Lyso\_activity \(\bigl[(\mathrm{LPC} + \mathrm{LPE}) - (\mathrm{PC} + \mathrm{PE})\bigr]\) exhibit a significant reduction (\(p<0.001\)), suggesting either inhibited phospholipase activity or accelerated removal of lysophospholipids to preserve membrane integrity, indicating an overall reduction in phospholipid turnover. This suggests inhibition of phospholipase A2 (PLA2) under stress conditions. Furthermore, it coincides with ↓DG/Phospho, indicating reduced phospholipid turnover.

\end{enumerate}

\paragraph{Physiological Implications}
\begin{enumerate}
  \item \textit{Conservation of membrane integrity.}  
        The notable reduction in lysophospholipids and overall Lyso\_activity suggests a decreased phospholipase activity for deacylation, thus safeguarding current bilayers from further damage due to simultaneous cold and nutrient scarcity.
\end{enumerate}

% ------------------------------------------------------------------
%  Combined canonical behaviour of LPC, LPE and their ratios 
% ------------------------------------------------------------------
\begin{table}[!ht]
  \begin{adjustwidth}{-2.25in}{0in}  % adjust as needed for your page
    \centering
    \caption{\bf Canonical responses of lysophospholipids (LPC, LPE) and their ratios under single abiotic stresses.}
    \label{table:lyso_combined}
    %
    % Part A: LPC & LPE pools
    %
    \subcaption*{A) Lysophospholipid Pools}
    \vspace{-0.5em}
    \begin{tabularx}{\textwidth}{@{} l l p{3.5cm} X c @{}}
      \toprule
      \textbf{Class} & \textbf{Stress} & \textbf{Response} & \textbf{Mechanism} & \textbf{Ref.} \\
      \midrule
      \multirow{3}{*}{\textbf{LPC}}
        & Low-P  & ↑ (lipase product) & PLA\textsubscript{2} activity rises to liberate phosphate; LPC is exported or rapidly reacylated. & [1] \\
        & Low-N  & ↓ / ≈              & Lipase activity suppressed to conserve N-rich PC; DAG backbone preferentially routed to TAG, not LPC. & [2] \\
        & Cold   & ↑ (acyl editing)   & Active PC ↔ LPC cycle inserts unsaturated acyl chains for cold acclimation. & [3] \\
      \addlinespace
      \multirow{3}{*}{\textbf{LPE}}
        & Low-P  & ↑ (lipase product) & PE hydrolysis releases phosphate; transient LPE spikes before re-acylation. & [4] \\
        & Low-N  & ↓ / ≈              & LPE formation limited by N scarcity; DAG diverted to storage lipids. & [5] \\
        & Cold   & ↑ (remodelling)    & PE ↔ LPE exchange supports insertion of unsaturated fatty acids. & [6] \\
      \bottomrule
    \end{tabularx}

    \vspace{1em}

    %
    % Part B: LPC/PC and LPE/PE ratios
    %
    \subcaption*{B) Lysophospholipid / Phospholipid Ratios}
    \vspace{-0.5em}
    \begin{tabularx}{\textwidth}{@{} l l p{3cm} X c @{}}
      \toprule
      \textbf{Ratio} & \textbf{Stress} & \textbf{Response} & \textbf{Interpretation} & \textbf{Ref.} \\
      \midrule
      \multirow{3}{*}{\(\tfrac{\mathrm{LPC}}{\mathrm{PC}}\)}
        & Low-P  & ↑            & Diagnostic of enhanced PLA\textsubscript{2}-mediated P scavenging. & [7] \\
        & Low-N  & ↓            & Choline head-group salvage limits LPC formation. & [8] \\
        & Cold   & ↑ (transient)& Rapid acyl editing before re-acylation restores PC pool. & [9] \\
      \addlinespace
      \multirow{3}{*}{\(\tfrac{\mathrm{LPE}}{\mathrm{PE}}\)}
        & Low-P  & ↑            & Reflects PE hydrolysis under P limitation. & [10] \\
        & Low-N  & ↓ / ≈        & Lipase repression conserves PE; LPE remains low or unchanged. & [11] \\
        & Cold   & ↑            & Enzymatic LPEAT-driven exchange introduces unsaturated chains. & [12] \\
      \bottomrule
    \end{tabularx}

    \vspace{0.5em}
    \begin{flushleft}
      {\footnotesize Arrows denote relative change vs.\ control (↑ increase, ↓ decrease, ≈ no consistent change).  
      Replace reference placeholders with appropriate citations.}
    \end{flushleft}
  \end{adjustwidth}
\end{table}



\subsection*{3. Carbon Sink and Allocation}

Under LI stress, carbon is progressively redirected from structural membranes into neutral storage lipids (Fig.~\ref{fig:lipid_remodeling} "Carbon Sink and Allocation"):

% ------------------------------------------------------------------
%  Canonical behaviour of triacylglycerol (TG) under single stresses
% ------------------------------------------------------------------
\begin{table}[!ht]
  \begin{adjustwidth}{-2.25in}{0in}  % Remove if the table already fits
    \centering
    \caption{\bf Canonical responses of triacylglycerol (TG) under single abiotic stresses.}
    \label{table:TG_responses}
    \begin{tabularx}{\textwidth}{@{} l l l X c @{}}
      \toprule
      \textbf{Class} & \textbf{Stress} & \textbf{Response} & \textbf{Mechanism (summary)} & \textbf{Ref.} \\
      \midrule
      \multirow{3}{*}{\textbf{TG}}
        & Low-P & ↑ / ≈ (species dependent) &
          Moderate TAG rise when excess DAG is *not* entirely channelled to DGDG;  
          P-starved cells deposit surplus carbon in TAG while conserving phospholipid-derived P. & [1] \\[0.4em]
        & Low-N & ↑↑ (strong induction) &
          Nitrogen shortage limits protein synthesis and sinks carbon into storage lipids;  
          PC breakdown supplies DAG that is rapidly acylated to TAG via DGAT/PDAT pathways. & [2] \\[0.4em]
        & Cold  & ↑ (energy buffer) &
          Cold stress slows growth and promotes TAG accumulation as a temporary carbon/energy reservoir;  
          unsaturated TAG species also scavenge free fatty acids released during membrane remodelling. & [3] \\
      \bottomrule
    \end{tabularx}
    \begin{flushleft}
      {\footnotesize Arrows denote relative change vs.\ control (↑ increase, ↓ decrease, ≈ no consistent change).  
      Replace reference placeholders with appropriate citations.}
    \end{flushleft}
  \end{adjustwidth}
\end{table}




\subsubsection*{3.1 Storage Lipid Reallocation}
\begin{enumerate}
  \item \textbf{Compartmentalized Carbon Redirection to Storage} \\
  TG/DG significantly increases (\(p<0.001\)), indicating a potent transformation of DG into TG, likely facilitated by DGAT enzymes external to membrane lipid recycling. Additionally, the TG/Galacto ratio \(\bigl[\mathrm{TG} - \tfrac{1}{2}(\mathrm{MGDG}+\mathrm{DGDG})\bigr]\) rises (\(p<0.001\)), suggesting that DG is directed away from galactolipids in chloroplasts to be stored. The DG\_to\_Gala value \(\bigl[\mathrm{DG} - \tfrac{1}{2}(\mathrm{MGDG}+\mathrm{DGDG})\bigr]\) also grows (\(p<0.001\)), reflecting an accumulation of DG relative to galactolipid pathways.

  \item \textbf{Storage over Phospholipid Maintenance}
  TG/Phospho \(\bigl[\mathrm{TG} - \tfrac{1}{2}(\mathrm{PC}+\mathrm{PE})\bigr]\) is significantly increased (\(p<0.001\)), suggesting that TG synthesis for storage is more prominent than the upkeep of membrane phospholipids, even with PC/PE conservation.
  
\end{enumerate}


\subsubsection*{3.2 Metabolic Trade-offs}
\begin{enumerate}
  \item \textbf{Photosynthetic Machinery Sacrificed for Storage} \\
  The metric \(\bigl[\tfrac{1}{2}(\mathrm{TG} + \mathrm{DG}) - \tfrac{1}{2}(\mathrm{MGDG} + \mathrm{DGDG})\bigr]\) shows a substantial increase (\(p<0.001\)), suggesting that carbon is being redirected from not only structural membranes but also from photosynthetic galactolipid reserves to TG storage.
  
  \item \textbf{Storage\_vs\_Membrane} \\
    \(\bigl[\tfrac{1}{2}(\mathrm{TG} + \mathrm{DG}) \;-\; \tfrac{1}{2}(\mathrm{PC} + \mathrm{PE})\bigr]\) rises significantly(\(p<0.001\)), showing net prioritization of carbon into storage lipids instead of major bilayer lipids.
  
\end{enumerate}

\paragraph{Physiological Implications}
\begin{enumerate}
  \item \textit{Energy reserve formation.}  The significant increase in TG synthesis, prioritizing over membrane lipids, suggests an adaptive strategy where carbon is redirected into energy reserves during cold and nutrient scarcity.
  \item \textit{Membrane down-regulation.}  A decrease in PC, PE, and PG compared to TAG/DG suggests ongoing membrane remodeling or partial disassembly to conserve resources necessary for survival.
  \item \textit{Dynamic lipid trade-off.}  These metrics together illustrate a coordinated switch from growth-promoting membranes toward stress-tolerant storage lipid pools.
\end{enumerate}


\subsection*{Integrated Lipid‐Metabolism Network, Ontology Enrichment, and Genetic Regulation of Triacylglycerol Homeostasis Indentifies a Triacylglycerol lipase}

As depicted in Figure \ref{fig:Fig6}A, our combined enzyme-reaction network, lipid enrichment analysis and genome-wide signals elucidates the interactions between DGs, MGs, TGs, and PEs. Three pivotal pathways are identified, as demonstrated in Table \ref{tab:linex_reactions}.

\begin{table}[!ht]
  \begin{adjustwidth}{-2.25in}{0in}
    \centering
    \caption{\textbf{Key enriched reactions in the LINEX2 sub‐network (low‐input vs.\ control)}}
    \label{tab:linex_reactions}
    \begin{tabularx}{\textwidth}{@{} l X X X @{}}
      \toprule
      \textbf{Pathway} & \textbf{Stoichiometry} & \textbf{Putative enzyme(s)} & \textbf{Interpretation (low‐input)} \\
      \midrule
      \multirow{3}{*}{\makecell[l]{\textbf{Lipolysis} \\ \textbf{axis}}} 
        & DG $\rightarrow$ MG + FA                  & LIPE‐like lipase                                   & Provides MG for re‐esterification or signalling.             \\
        & MG + FA $\rightarrow$ TG                  & \textit{PNPLA3} (triacylglycerol synthase)         & \textbf{$\downarrow$ Flux} — storage synthesis suppressed.  \\
        & TG $\rightarrow$ DG + FA                  & \textit{PNPLA1} (SORBI\_3001G041900)               & \textbf{$\uparrow$ Lipolysis} — dominant driver of DG pool. \\
      \addlinespace
      \makecell[l]{\textbf{Phospholipid} \\ \textbf{recycling}} 
        & PE + DG $\rightleftharpoons$ LPE + TG     & LRO1‐type acyl‐transferase                         & Membrane PE shuttles acyl chains to TG.                     \\
      \addlinespace
      \multirow{3}{*}{\makecell[l]{\textbf{Alternative} \\ \textbf{TG conversions}}}
        & PE $\rightarrow$ DG + PI (acyl transfer)  & EPTB (phosphoethanolamine phosphotransferase)      & Recycles PE headgroups into TG $\Rightarrow$ DG cascade.     \\
        & TG $\rightarrow$ RHEA:32843               & Unspecified lipase                                 & Alternative TG hydrolysis branch.                           \\
        & LPC $\leftrightarrow$ DG                  & Unspecified transferase                            & LPC $\leftrightarrow$ DG interconversion at droplet surface. \\
      \bottomrule
    \end{tabularx}
  \end{adjustwidth}
\end{table}


The conversion of TG to DG catalyzed by PNPLA1 is identified as a critical process, complemented by the transformation from DG to TG facilitated by PNLPLA3. This enzymatic relationship suggests that modifications in TG turnover via PNPLA1 could account for the observed variations in lipid profiles (Fig X) when contrasting LI conditions to the C.

Additionally, the LION lipid-ontology enrichment analysis (Figure \ref{fig:Fig6}B) identifies significant terms such as "triacylglycerols [GL0301]" (FDR = 0.002), "lipid storage" (FDR = 0.008), "lipid droplet" (FDR = 0.008), "headgroup with neutral charge" (FDR = 0.012), and "glycerolipids [GL]" (FDR = 0.015) at a significance threshold of (−log₁₀ q > 1.3). These enriched terms suggest the accumulation and mobilization of TG pools, aligning with the biochemical roles of PNPLA1 and PNPLA3 in the recycling of triglycerides and diglycerides (TG~$\leftrightarrow$~DG):

\begin{array}{rcl}
\text{TG} & 
  \overset{\text{PNPLA3}}{\underset{\text{PNPLA1}}{\rightleftharpoons}} & 
\text{DG} \quad \xrightarrow{\text{}} \quad \text{TG/DG recycling}
\end{array}
\]

Furthermore, five independent GWAS results for TGs (Figure \ref{fig:Fig6}C) consistently identify a single locus on chromosome 1 annotated as the triacylglycerol lipase gene SORBI\_3001G041900 (SbPNPLA1) above the pvalue cutoffe Sum\_TG, TG\_54\_6, TG\_54\_7, TG\_56\_4, and TG\_56\_6 plots. The recurrent emergence of this peak across both aggregate and individual species traits highlights the crucial role of PNPLA1 in triacylglycerol homeostasis. Collectively, these findings support a model wherein genetic variation at the PNPLA1 locus directs the variability in lipid storage and mobilization, especially under cold and nutrient-deficient conditions.


\begin{figure}[htbp]
  \centering
  \includegraphics[width=\textwidth]{fig/main/Fig6.png}
  \caption{\textbf{Integrated lipid‐metabolism network, ontology enrichment, and GWAS of triacylglycerol traits.}
    \textbf{(A)} Enzyme‐reaction network assembled in \texttt{LaTeX2Enrich}, showing diacylglycerols (DGs), monoacylglycerols (MGs), and triacylglycerols (TGs) as nodes.  Edges denote biochemical conversions: DG → MG (LIPE), MG → TG (PNLPLA3), TG → DG (PNPLA1, highlighted), LPC → DG, DG → LPC, TG → RHEA:32843, PE → DG (EPTB), and PE/DG → LPE (LRO1).  The prominent PNPLA1‐catalyzed TG→DG reaction (blue‐outlined node) suggests a key role in our low‑input vs.\ control comparison.
    \textbf{(B)} LION lipid‐ontology enrichment results.  The top enriched categories (−log\textsubscript{10}(FDR $q$) > 2, red dashed line) are “triacylglycerols [GL0301]”, “lipid storage”, “lipid droplet”, “headgroup with neutral charge” and “glycerolipids [GL]”.  This concords with our GWAS hits at PNPLA1/PNLPLA3, both involved in TG ↔ DG cycling.
    \textbf{(C)} Multi‑track Manhattan plots (CMplot, \texttt{multracks=TRUE}) for five traits (Sum\_TG, TG\_54\_6, TG\_54\_7, TG\_56\_4, TG\_56\_6).  Chromosome 1 (green vertical line) harbors a cluster of highly significant SNPs (–log\textsubscript{10} $p$ ≥ 7, red dots) at the triacylglycerol lipase locus SORBI\_3001G041900 (annotated).  All five traits share this peak, underscoring the importance of TG→DG hydrolysis in modulating lipid levels under low‑input vs.\ control conditions.}
  \label{fig:Fig6}
\end{figure}





%--------------------------------------------------------------------
\bibliographystyle{plainnat}
\bibliography{lipid_refs}



\section*{Discussion}
Something something lipids are good. 


\section*{Conclusion}

because we can For more information, see \nameref{S1_Appendix}.

\section*{Supporting information}

% Include only the SI item label in the paragraph heading. Use the \nameref{label} command to cite SI items in the text.
\paragraph*{S1 Fig.}
\label{S1_Fig}
{\bf Bold the title sentence.} Add descriptive text after the title of the item (optional).

\paragraph*{S2 Fig.}
\label{S2_Fig}
{\bf Lorem ipsum.} Maecenas convallis mauris sit amet sem ultrices gravida. Etiam eget sapien nibh. Sed ac ipsum eget enim egestas ullamcorper nec euismod ligula. Curabitur fringilla pulvinar lectus consectetur pellentesque.

\paragraph*{S1 File.}
\label{S1_File}
{\bf Lorem ipsum.}  Maecenas convallis mauris sit amet sem ultrices gravida. Etiam eget sapien nibh. Sed ac ipsum eget enim egestas ullamcorper nec euismod ligula. Curabitur fringilla pulvinar lectus consectetur pellentesque.

\paragraph*{S1 Video.}
\label{S1_Video}
{\bf Lorem ipsum.}  Maecenas convallis mauris sit amet sem ultrices gravida. Etiam eget sapien nibh. Sed ac ipsum eget enim egestas ullamcorper nec euismod ligula. Curabitur fringilla pulvinar lectus consectetur pellentesque.

\paragraph*{S1 Appendix.}
\label{S1_Appendix}
{\bf Lorem ipsum.} Maecenas convallis mauris sit amet sem ultrices gravida. Etiam eget sapien nibh. Sed ac ipsum eget enim egestas ullamcorper nec euismod ligula. Curabitur fringilla pulvinar lectus consectetur pellentesque.

\paragraph*{S1 Table.}
\label{S1_Table}
{\bf Lorem ipsum.} Maecenas convallis mauris sit amet sem ultrices gravida. Etiam eget sapien nibh. Sed ac ipsum eget enim egestas ullamcorper nec euismod ligula. Curabitur fringilla pulvinar lectus consectetur pellentesque.

\section*{Acknowledgments}
Me, myself and I

\nolinenumbers

% Either type in your references using
% \begin{thebibliography}{}
% \bibitem{}
% Text
% \end{thebibliography}
%
% or
%
% Compile your BiBTeX database using our plos2015.bst
% style file and paste the contents of your .bbl file
% here. See http://journals.plos.org/plosone/s/latex for 
% step-by-step instructions.
% 
\begin{thebibliography}{10}


\bibitem[Dall’Osto \emph{et~al.}(2012)]{DallOsto2012}
Dall’Osto, L., Cazzaniga, S., Bressan, M., Paleček, D., Židek, K., Jennings, R.~C., \& Bassi, R. (2012).  
Zeaxanthin protects plant photosynthesis by modulating chlorophyll triplet yield in specific light‐harvesting antenna subunits.  
\emph{Journal of Biological Chemistry}, 287(10), 6180–6190.

\bibitem[Demmig‐Adams \emph{et~al.}(2020)]{DemmigAdams2020}
Demmig‐Adams, B., Adams, W.~W., III, \& Holzwarth, A.~R. (2020).  
Zeaxanthin, a molecule for photoprotection in many different environments.  
\emph{Molecules}, 25(1), 100.

\bibitem[Guardini \emph{et~al.}(2020)]{Guardini2020}
Guardini, Z., Bressan, M., Caferri, R., Bassi, R., \& Dall’Osto, L. (2020).  
Identification of a pigment cluster catalysing fast photoprotective quenching response in CP29.  
\emph{Nature Plants}, 6, 1261–1273.


\bibitem[Williams and Morgan(1979)]{Williams1979}
Williams, E.~A., \& Morgan, P.~W. (1979).  
Floral initiation in sorghum hastened by gibberellic acid and far‐red light.  
\emph{Planta}, 145, 269–272.

\bibitem[Lee \emph{et~al.}(1998)]{Lee1998}
Lee, I.~J., Foster, K.~R., \& Morgan, P.~W. (1998).  
Photoperiod control of gibberellin levels and flowering in sorghum.  
\emph{Plant Physiology}, 116, 1003–1011.

\bibitem[Paul \emph{et~al.}(2020)]{Paul2020}
Paul, P., Dhatt, B.~K., Miller, M., \emph{et~al.} (2020).  
MADS78 and MADS79 are essential regulators of early seed development in rice.  
\emph{Plant Physiology}, 182, 933–948.

\bibitem[Jabir and Mahmoud(2021)]{Jabir2021}
Jabir, D.~A.~A., \& Mahmoud, M.~R. (2021).  
The effect of temperature stress associated with different planting dates and levels of gibberellic acid on the growth of sorghum spring.  
\emph{IOP Conference Series: Earth and Environmental Science}, 923, 012090.

\bibitem[Young and Britton(1989)]{Young1989}
Young, A.~J., \& Britton, G. (1989).
\newblock The distribution of alpha-carotene in the photosynthetic pigment-protein complexes of higher plants.
\newblock \emph{Plant Science}, 64, 179–183.

\bibitem[Vishwakarma \emph{et~al.}(2015)]{Vishwakarma2015}
Vishwakarma, A., Tetali, S.~D., Selinski, J., Scheibe, R., \& Padmasree, K. (2015).
\newblock Importance of the alternative oxidase (AOX) pathway in regulating cellular redox and ROS homeostasis to optimize photosynthesis during restriction of the cytochrome oxidase pathway in \emph{Arabidopsis thaliana}.
\newblock \emph{Annals of Botany}, 116(4), 553–566.
\newblock \doi{10.1093/aob/mcv066}

\bibitem[Gandin \emph{et~al.}(2014)]{Gandin2014}
Gandin, A., Dinakar, C., McDonald, A.~E., \& Vanlerberghe, G.~C. (2014).
\newblock Cooperation between the AOX pathway and nitrate assimilation to maintain optimal photosynthesis by regulating the accumulation of reducing equivalents.
\newblock \emph{Plant Physiology}.
  
\bibitem[Sayeed \emph{et~al.}(2016)]{Sayeed2016}
Sayeed, M.~S.~B., \emph{et~al.} (2016).  
Critical analysis on characterization, systemic effect, and therapeutic potential of beta‑sitosterol.  
\emph{Medicines}.

\bibitem[Mueller and Brown(1980)]{Mueller1980}
Mueller, S.~C., \& Brown, R.~M., Jr. (1980).  
Evidence for an intramembrane component associated with a cellulose microfibril synthesizing complex in higher plants.  
\emph{Journal of Cell Biology}, 84(3), 315–326.

\bibitem[Somerville(2006)]{Somerville2006}
Somerville, C. (2006).  
Cellulose synthesis in higher plants.  
\emph{Annual Review of Cell and Developmental Biology}, 22, 53–78.

\bibitem[Hu \emph{et~al.}(2018)]{Hu2018}
Hu, H., \emph{et~al.} (2018).  
Cellulose synthase mutants distinctively affect cell growth and cell wall integrity for plant biomass production in Arabidopsis.  
\emph{Plant Cell Physiology}, 59(6), 1142–1154.

\bibitem[Arioli \emph{et~al.}(1998)]{Arioli1998}
Arioli, T., \emph{et~al.} (1998).  
Molecular analysis of cellulose biosynthesis in Arabidopsis.  
\emph{Science}, 279, 717–720.

\bibitem[Persson \emph{et~al.}(2007)]{Persson2007}
Persson, S., \emph{et~al.} (2007).  
Genetic evidence for three unique components in primary cell‐wall cellulose synthase complexes.  
\emph{Proceedings of the National Academy of Sciences USA}, 104(39), 15566–15571.

\bibitem[Cano‐Delgado \emph{et~al.}(2003)]{CanoDelgado2003}
Cano‐Delgado, A.~I., \emph{et~al.} (2003).  
Reduced cellulose synthesis invokes lignification and defense responses in Arabidopsis thaliana.  
\emph{Plant Journal}, 34(4), 351–362.

\bibitem[Hernández‐Blanco \emph{et~al.}(2007)]{HernandezBlanco2007}
Hernández‐Blanco, C., \emph{et~al.} (2007).  
Impaired cellulose synthesis enhances disease resistance in Arabidopsis.  
\emph{Plant Cell}, 19(3), 890–903.

\bibitem[Tomlinson \emph{et~al.}(2004)]{Tomlinson2004}
Tomlinson, K., \emph{et~al.} (2004).  
Effects of inhibiting cellulose biosynthesis on nitrogen, phosphorus, and sulfur metabolism in Brassica napus.  
\emph{Plant Physiology}, 134(2), 568–577.

\bibitem[Ekman \emph{et~al.}(2008)]{Ekman2008}
Ekman, D., \emph{et~al.} (2008).  
Carbon partitioning during secondary wall biosynthesis in Arabidopsis stems.  
\emph{Plant Journal}, 53(3), 425–436.

\bibitem[Iyer \emph{et~al.}(2008)]{Iyer2008}
Iyer, P.~V.~V., \emph{et~al.} (2008).  
Alteration of cellulose and lignin in Arabidopsis via RNAi of cellulose synthase genes.  
\emph{Molecular Plant}, 1(2), 212–220.

\bibitem[Shi \emph{et~al.}(2012)]{Shi2012}
Shi, D., \emph{et~al.} (2012).  
Genetic analysis of Arabidopsis cellulose mutants reveals secondary cell wall defects.  
\emph{Plant Physiology}, 158(4), 1587–1595.

\bibitem[Tan \emph{et~al.}(2011)]{Tan2011}
Tan, J., \emph{et~al.} (2011).  
Enhancing seed protein content by down‐regulating cellulose synthesis in rice.  
\emph{Plant Biotechnology Journal}, 9(7), 834–842.

\bibitem[Yoshie‐Stark \emph{et~al.}(2008)]{YoshieStark2008}
Yoshie‐Stark, Y., \emph{et~al.} (2008).  
Manipulation of cell wall composition to increase seed protein content in maize.  
\emph{Journal of Agricultural Food Chemistry}, 56(11), 3981–3988.

\bibitem[Knowles(1983)]{Knowles1983}
Knowles, N.~R. (1983).  
Carbohydrate and protein accumulation during seed development in peas.  
\emph{Plant Physiology}, 72(1), 45–50.

\bibitem[Hu \emph{et~al.}(2020)]{Hu2020}
Hu, H., \emph{et~al.} (2020).  
Manipulating cellulose synthase for seed storage protein improvement.  
\emph{Plant Cell Reports}, 39(5), 607–619.


\bibitem{Yu2002}
Yu, B., Xu, C., \& Benning, C. (2002).  
\emph{Arabidopsis disrupted in SQD2 encoding sulfolipid synthase is impaired in phosphate-limited growth.}  
\textit{Proceedings of the National Academy of Sciences USA}, 99, 5732–5737.

\bibitem{Sun2021}
Sun, Y., Song, K., Liu, L., \emph{et al.} (2021).  
\emph{Sulfoquinovosyl diacylglycerol synthase 1 impairs glycolipid accumulation and photosynthesis in phosphate-deprived rice.}  
\textit{Journal of Experimental Botany}, 72(18), 6510–6523.

\bibitem{Qin2015}
Qin, X., Suga, M., Kuang, T., \& Shen, J. R. (2015).  
\emph{Structural basis for energy transfer pathways in the plant PSI-LHCI supercomplex.}  
\textit{Science}, 348, 989–995.

\bibitem{Umena2011}
Umena, Y., Kawakami, K., Shen, J. R., \& Kamiya, N. (2011).  
\emph{Crystal structure of oxygen-evolving photosystem II at 1.9 Å resolution.}  
\textit{Nature}, 473, 55–60.

\bibitem{YuBenning2003}
Yu, B., \& Benning, C. (2003).  
\emph{Anionic lipids are required for chloroplast structure and function in Arabidopsis.}  
\textit{The Plant Journal}, 36, 762–770.

\bibitem{Essigmann1998}
Essigmann, B., Güler, S., Narang, R. A., Linke, D., \& Benning, C. (1998).  
\emph{Phosphate availability affects thylakoid lipid composition and the expression of SQD1, a gene required for sulfolipid biosynthesis in Arabidopsis thaliana.}  
\textit{Proceedings of the National Academy of Sciences USA}, 95, 1950–1955.

\bibitem{Nakamura2013}
Nakamura, Y. (2013).  
\emph{Phosphate starvation and membrane lipid remodeling in seed plants.}  
\textit{Progress in Lipid Research}, 52, 43–50.

\bibitem{Yang2011}
Yang, Y., Yu, X., Song, L., \& An, C. (2011).  
\emph{ABI4 Activates DGAT1 Expression in Arabidopsis Seedlings during Nitrogen Deficiency.}  
\textit{Plant Physiology}, 156(2), 874–883.

\bibitem{Tan2018}
Tan, W.-J., Yang, Y.-C., Zhou, Y., Huang, L.-P., Xu, L., Chen, Q.-F., Yu, L.-J., \& Xiao, S. (2018).  
\emph{DIACYLGLYCEROL ACYLTRANSFERASE and DIACYLGLYCEROL KINASE Modulate Triacylglycerol and Phosphatidic Acid Production in the Plant Response to Freezing Stress.}  
\textit{Plant Physiology}, 177(4), 1304–1316.

\bibitem{Zhang2009}
Zhang, M., Fan, J., Taylor, D. C., \& Ohlrogge, J. B. (2009).  
\emph{DGAT1 and PDAT1 Acyltransferases Have Overlapping Functions in Arabidopsis Triacylglycerol Biosynthesis and Are Essential for Normal Pollen and Seed Development.}  
\textit{The Plant Cell}, 21(12), 3885–3901.

\end{thebibliography}

%==========================================
%   Start the Supplementary Material section
%==========================================
\FloatBarrier
\section*{Supplementary Material}
\beginsupplement


%========================================================
%  Supplementary Figure S1 – TIC traces
%========================================================
\begin{figure}[htp]
  \centering
  \includegraphics[width=\textwidth]{fig/supp/SuppFig1.png}
  \caption{
    Total ion current (TIC) traces for all injections in the lipidomics run. 
    {\bf(A)} Control samples (top panel). 
    {\bf(B)} Lowinput samples (bottom panel).
  }
  \label{fig:S1}
\end{figure}



%========================================================
%  Supplementary Figure S2 - SERFF RSD & PCA results
%========================================================
\begin{figure}[htp]
  \centering
  % Combined panel A (left) and B (right) in one image file
  \includegraphics[width=\textwidth]{fig/supp/SuppFig2.png}
  \caption{
    SERRF-normalized quality metrics across injections. 
    {\bf(A)} Control samples: per-feature RSD distributions (boxplot) and PCA of QC vs.\ biological samples. 
    {\bf(B)} Low-Input samples: same metrics after SERRF correction. 
    Red and green dots represents blank and QC respectively. Both panels demonstrate tight RSDs and clear separation of QC from biological samples in PC1/PC2.
  }
  \label{fig:S2}
\end{figure}








%========================================================
%  Supplementary Figure 3: Spatial Analysis 
%========================================================
\begin{figure}[htp]
  \centering
  \includegraphics[width=\textwidth]{fig/supp/SuppFig3.png}
  \caption{
    Spatial‐analysis diagnostics for the lipid TG(10:0/10:0/10:0) from the SpATS model. 
    {\bf(A)} Control: 3D spatial‐trend surface (row vs.\ column displacement). 
    {\bf(B)} Control diagnostics: (i) raw data, (ii) fitted values, (iii) residuals, (iv) fitted spatial trend, (v) genotypic BLUPs, (vi) histogram of BLUPs. 
    {\bf(C)} Low‐Input: 3D spatial‐trend surface. 
    {\bf(D)} Low‐Input diagnostics, as in (B).
  }
  \label{fig:S3}
\end{figure}





%\begin{figure}[htp]
%  \centering

  % ---------- row 1 ----------
  %\begin{subfigure}[t]{0.48\textwidth}
  %  \includegraphics[width=\linewidth]{fig/supp/SuppFig_3A_Lipid_Counts.png}
  %  \caption{Number of lipid \textit{species}.}
  %  \label{fig:S3A}
  %\end{subfigure}\hfill
  %\begin{subfigure}[t]{0.48\textwidth}
  %  \includegraphics[width=\linewidth]{fig/supp/SuppFig_3B_trad_nontrad_counts.png}
  %  \caption{Number of lipid \textit{classes}.}
  %  \label{fig:S3B}
  %\end{subfigure}

  %\vspace{1em}

  % ---------- row 2 (centred) ----------
  %\begin{subfigure}[t]{0.55\textwidth}
  %  \centering
  %  \includegraphics[width=\linewidth]%{fig/supp/SuppFig_3C_Lipid_Overlap_Venn_traditional.png}
    %\caption{Shared and unique lipid species.}
    %\label{fig:S3C}
  %\end{subfigure}

  %\caption{Overview of lipid coverage in Control and Low-Input samples.}
  %\label{fig:S3}
%\end{figure}



%========================================================
% Supplementary Figure S4 - Lipid species/class count   (panels A, B, C, and D)
%========================================================
\begin{figure}[htp]
  \centering
  \includegraphics[width=\textwidth]{fig/supp/SuppFig4.png}
  \caption{
    Summary of lipid coverage in Control vs.\ Lowinput runs. 
    {\bf(A)} Number of detected lipid species per major class.
    {\bf(B)} Breakdown of species counts within the two largest classes: Glycerolipid and Glyverophospholipid.
    {\bf(C)} Venn diagram of all lipid species: 184 species (∼58.2\%) are shared, while Control only (65, 20.6\%) and Lowiput only (67, 21.2\%) show a small number of unique detections in each run.
  }
  \label{fig:S4}
\end{figure}


%========================================================
% Supplementary Figure S5 - TIC - Class, Glycerolipid, Glycerophospholipid
%========================================================
\begin{figure}[htp]
  \centering
  \includegraphics[width=\textwidth]{fig/supp/SuppFig5.png}
  \caption{
    Distribution of total ion current (TIC) by lipid category in Control vs.\ Lowinput samples. 
    {\bf(A)} Major lipid classes as \% of TIC: glycerolipids dominate (~64 \% Control, ~52 \% Lowinput), followed by glycerophospholipids (~30 \% vs 25 \%). Sphingolipids sees the highest change.
    {\bf(B)} Breakdown of the glycerolipid pool: mono- and di-galactosyldiacylglycerols, diacylglycerols and triacylglycerols together account for most of the glycerolipid TIC, with nearly identical subclass proportions in both runs. 
    {\bf(C)} Breakdown of the glycerophospholipid pool: phosphatidylcholines (>79 \%), phosphatidylethanolamines (~18 \%), and minor head-groups (LPC, PS, etc.) together explain the glycerophospholipid TIC. The results are very consistent between Control and Lowinput.
  }
  \label{fig:S5}
\end{figure}



%========================================================
%  Supplementary Figure S4 - Lipid ratio contrasts under low-P
%========================================================
\begin{figure}[htp]
  \centering
  % Adjust width fraction as needed (e.g., 0.8\textwidth or \textwidth)
  \includegraphics[width=0.8\textwidth]{fig/supp/SuppFig_4_lipid_ratio_linear_lowP.png}
  \caption{$\Delta$Z-score contrasts for lipids under LI. 
    The panel shows violin+boxplots for metrics such as \textitt{SQDG-Spared}, \textitt{LPC-PC}, and \textitt{LPE-PE} under Control versus LowInput conditions. 
    Stars denote significance levels (***: $p<0.001$, **: $p<0.01$, *: $p<0.05$) from appropriate statistical tests. 
    A negative $\Delta$Z in SQDG\_Spared indicates sulfolipid is not upregulated relative to galactolipids and PG; 
    \textitt{LPC-PC} is not significantly changed, whereas \textitt{LPE-PE} and composite Lyso\_activity shift toward values consistent with selective PE deacylation.}
  \label{fig:S4_lipid_ratio_lowP}
\end{figure}

%========================================================
%  Supplementary Figure S5 - TIC Proportions for LPC and LPE
%========================================================
\begin{figure}[htp]
  \centering
  % Adjust width fraction as appropriate, e.g., 0.6\textwidth or \textwidth
  \includegraphics[width=0.7\textwidth]{fig/supp/SuppFig_5_TIC_LPC_LPE.png}
  \caption{Total ion current (TIC) proportions of lysophosphatidylcholine (LPC) and lysophosphatidylethanolamine (LPE) under Control and Low-P conditions. The plot displays relative TIC share of LPC versus LPE; stars denote significance levels (e.g., *: $p<0.05$, **: $p<0.01$) from appropriate tests. An increase in the LPE fraction and corresponding decrease in LPC under low-P suggests selective deacylation of PE for P salvage, while PC-derived LPC remains relatively stable.}
  \label{fig:S5_TIC_LPC_LPE}
\end{figure}



\end{document}


